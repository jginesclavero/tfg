\chapter{Introducción}
\label{cap:introduccion}

\section{Robótica móvil}
\label{cap:roboticamovil}
Los robots móviles son máquinas con la capacidad de desplazarse por un entorno. Para ello hacen uso de un sistema de automoción, ya sean ruedas, patas u orugas. 
La robótica móvil está sufriendo un gran crecimiento por el abaratamiento del hardware y las grandes oportunidades que nos ofrecen, ya sea en materia de educación o en materia industrial. 
Cuando nos encontramos con un problema en el que un robot móvil puede ser la mejor solución tendremos que tener en cuenta los siguientes problemas a solucionar:
\begin{itemize}
\item \textbf{Percepción} Debemos resolver el problema de la percepción. Para ello equiparemos a nuestro robot de sensores, como puede ser una cámara, un láser, sensores de odometría, bumpers para obtener información acerca de que hay en el escenario en el que se encuentra el robot.
\item \textbf{Localización} Una vez resuelto el problema de conocer que hay cerca del robot, necesitamos conocer la posición del robot dentro del escenario. Esto se puede resolver con algoritmos como MonteCarlo, con cámaras cenitales, tiangulación con balizas, etc. Para resolver la localización también es común el uso de mapas. 
\item \textbf{Navegación} La característica principal de los robots móviles es que tienen la capacidad de desplazarse, pero la navegación no solo se cumple cuando el robot se mueve, el robot debe moverse a un punto determinado, de una forma segura, es decir sin chocar con los objetos del escenario, y de la forma más eficaz posible. Dentro de la navegación se pueden distinguir 2 partes, la navegación local que se ocupa del movimiento del robot y de evitar obstáculos y la navegación global qué establece rutas.
\item \textbf{Inteligencia} El siguiente problema se encuentra en un nivel de abstracción más alto. El problema de la inteligencia se refiere a qué tiene que hacer el robot, qué finalidad debe cumplir.
\item \textbf{Autonomía} Habrá momentos en los que el robot deba tomar algunas decisiones, en referencia a su estado en el escenario, por ejemplo si se acerca mucho a una pared, o si llega a su destino correctamente.
\item \textbf{Interacción con los humanos} Los robots suelen crearse para facilitarnos una tarea o asistirnos en nuestro día a día, por esto los robots deben contar con una interfaz hombre-maquina con la que comunicarnos cosas o con la que nosotros podamos ayudar al robot para así mejorar sus acciones.
\end{itemize}

\section{Mapeado}
\label{cap:mapeado}

\section{RoboCup@Home}
\label{cap:robocup}

\section{Estructura de la memoria}
\label{cap:estructuradelamemoria}


