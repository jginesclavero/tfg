\chapter{Experimentación}
\label{cap:experimentacion}
En este capitulo desarrollaremos e ilustraremos los diferentes experimentos o pruebas unitarias que se han realizado para la validación del sistema de mapeado dinámico y el sistema de navegación semántica. En primer lugar se realizarán test de mapeado en un entorno doméstico. Con esto comprobaremos que los objetos son eliminados del mapa si desaparecen y como el mapa va cambiando. También haremos pruebas en una estancia de la casa que no aparece en los mapas y comprobaremos como se añade correctamente al mapa. En segundo lugar haremos pruebas de navegación en un entorno doméstico y dinámico y por ultimo expondremos los experimentos llevados a cabo en la RoboCup@Home en el transcurso de la competición.

\section {Mapeado en entorno doméstico}
\label{cap:mapeadodomestico}

pruebas unitarias. 


La figura \ref{fig:initserver} fue captada al iniciar el algoritmo. Observamos que la mayor parte del mapa de corto plazo se encuentra en una posición de desconocimiento y que se han ido incluyendo en este las zonas libres, las paredes y la estantería. Los puntos morados y verdes corresponden a la representación de las muestras tomadas por el láser.

\begin{figure}[H]
  \begin{center}
    \subfigure[]{\includegraphics[width=5cm,height=5cm]{img/cap7/incrementmap}}
    \subfigure[]{\includegraphics[width=5cm,height=5cm]{img/cap7/incrementmap2}}
    \subfigure[]{\includegraphics[width=5cm,height=5cm]{img/cap7/incrementmap-rviz}}
  \end{center}
  \caption{Visión del simulador, (a) y (b), y mapa a corto plazo (c).}
  \label{fig:initserver}
\end{figure}

Tras el inicio del algoritmo se añadió un objeto nuevo al escenario. Esto se representa en la figura \ref{fig:includeobject}. Vemos como el algoritmo añade el objeto al mapa y lo sitúa en una posición coherente respecto a la posición que ocupa el objeto en el escenario simulado.

\begin{figure} [H]
  \begin{center}
    \includegraphics[width=6cm,height=5cm]{img/cap7/incrementmap-object3}
    \includegraphics[width=6cm,height=5cm]{img/cap7/incrementmap-object}
  \end{center}
  \caption{Añadimos un objeto al escenario}
  \label{fig:includeobject}
\end{figure}

Una vez que el algoritmo a incluido el objeto en el mapa procedemos a eliminarlo del escenario simulado. Esto se representa en la figura  \ref{fig:deleteobject}. Vemos como el algoritmo ha comenzado a borrar el objeto, por lo que el valor de las celdas que estaban ocupadas por el objeto ahora es mucho menor.

\begin{figure}[H]
  \begin{center}
    \includegraphics[width=6cm,height=5cm]{img/cap7/incrementmap2}
    \includegraphics[width=6cm,height=5cm]{img/cap7/incrementmap-object2}
  \end{center}
  \caption{Eliminamos un objeto del escenario}
  \label{fig:deleteobject}
\end{figure}

\section {Navegación con obstáculos dinámicos}
\label{cap:navegacionconobstaculos}

\section {Experimentación en la Robocup}
\label{cap:experimentacionrobocup}




LARGO PLAZO

\begin{figure}[H]
  \begin{center}
    \includegraphics[width=10cm,height=6cm]{img/cap7/addingobject-gazebo}
    \includegraphics[width=10cm,height=5cm]{img/cap7/addingobject-longmap}
  \end{center}
  \caption{Añadimos un objeto al mapa de largo plazo}
  \label{fig:addobjectlongmap}
\end{figure}

\begin{figure}[H]
  \begin{center}
    \includegraphics[width=10cm,height=5cm]{img/cap7/deletingobject-gazebo}
    \includegraphics[width=10cm,height=5cm]{img/cap7/deletingobject-longmap}
  \end{center}
  \caption{Borramos un objeto del mapa de largo plazo}
  \label{fig:deleteobjectlongmap}
\end{figure}


EXP ZONAS DESCONOCIDAS


\begin{figure}[hbtp]
  \begin{center}
    \includegraphics[width=12cm,height=7cm]{img/cap7/grannieAnne-ext}
  \end{center}
  \caption{Escenario extendido}
  \label{fig:grannieAnne-ext}
\end{figure}

\begin{figure}[hbtp]
  \begin{center}
    \includegraphics[width=10cm,height=6cm]{img/cap7/localization-ext}
  \end{center}
  \caption{Detalle de la localización}
  \label{fig:localization-ext}
\end{figure}


\begin{figure}[hbtp]
  \begin{center}
    \subfigure[Mapa total]{\includegraphics[width=12cm,height=7cm]{img/cap7/map-ext}}
    \subfigure[Mapa de largo plazo]{\includegraphics[width=12cm,height=7cm]{img/cap7/longmap-ext}}
    \subfigure[Mapa de corto plazo]{\includegraphics[width=12cm,height=7cm]{img/cap7/shortmap-ext}}
  \end{center}
  \caption{Mapas del escenario extendido}
  \label{fig:maps-ext}
\end{figure}

En la figura \ref{fig:grannieAnne-ext} vemos como el escenario ha sido extendido, añadiéndole unos pasillos a la derecha de la casa. El mapa estático usado es el mismo que el mostrado en la figura \ref{fig:mapaestatico}. Vemos como después de recorrer la parte desconocida del mapa se ha añadido al mapa de largo plazo y también al mapa total, figura \ref{fig:maps-ext}. Observamos en las flechas rojas bajo el robot que la incertidumbre en la posición del robot es mínima, figura \ref{fig:localization-ext}. 
