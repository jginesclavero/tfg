\documentclass[twoside,a4paper,12pt,openany]{book}

\usepackage[utf8]{inputenc}
\usepackage[spanish]{babel}
\usepackage{geometry}
\usepackage{makeidx}
\usepackage{graphicx}
\usepackage{subfig}
%\usepackage[Bjornstrup]{fncychap}
\usepackage{a4wide}
\usepackage{named}
\usepackage[table]{xcolor}
\usepackage{listings}
\usepackage{pdfpages}
\usepackage{appendix}
\usepackage{amsmath}
\usepackage{amsfonts}

% From texlive-science
\usepackage{algorithm}
\usepackage{algorithmic}

%\DeclareGraphicsExtensions{.jpg,.pdf,.png,.ps}

\usepackage{color}
\definecolor{gray97}{gray}{.97}
\definecolor{gray75}{gray}{.75}
\definecolor{gray45}{gray}{.45}

\pagestyle{headings}
\geometry{a4paper, left=3.5cm, right=2cm, top=3cm, bottom=2cm, headsep=1.5cm}

\widowpenalty=10000
\clubpenalty=10000
\hyphenpenalty=10000
\tolerance=10000
%Para que no corte las palabras al final de la linea.

% Latex command para crear listas sin espacio entre
% un item y otro.
\newenvironment{packed_item}{
\begin{itemize}
  \setlength{\itemsep}{1pt}
  \setlength{\parskip}{0pt}
  \setlength{\parsep}{0pt}
}{\end{itemize}}

% Latex command para crear listas umeradas sin 
% espacio entre un item y otro.
\newenvironment{packed_enum}{
\begin{enumerate}
  \setlength{\itemsep}{1pt}
  \setlength{\parskip}{0pt}
  \setlength{\parsep}{0pt}
}{\end{enumerate}}

% Configuracion del entorno lstlistings
\lstset{
  frame=Ltb,
  framerule=0pt,
  aboveskip=0.5cm,
  xleftmargin=0.7cm,
  framextopmargin=3pt,
  framexbottommargin=3pt,
  framexleftmargin=0.4cm,
  framesep=0pt,
  rulesep=.4pt,
  backgroundcolor=\color{gray97},
  rulesepcolor=\color{black},
  %
  stringstyle=\ttfamily,
  showstringspaces = false,
  basicstyle=\small\ttfamily,
  commentstyle=\color{gray45},
  keywordstyle=\bfseries,
  %
  numbers=left,
  numbersep=15pt,
  numberstyle=\tiny,
  numberfirstline = false,
  breaklines=true,
}

% minimizar fragmentado de listados
\lstnewenvironment{listing}[1][]
{\lstset{#1}\pagebreak[0]}{\pagebreak[0]}

\lstdefinestyle{consola}
{basicstyle=\scriptsize\bf\ttfamily,
backgroundcolor=\color{gray75},
}

\lstdefinestyle{C}
{language=C,
}

\lstdefinestyle{sh}
{language=sh,
}

\makeindex

\begin{document}

\thispagestyle{empty}

\baselineskip 1.35\baselineskip

\vspace{2cm}

\begin{figure}[htb]
  \centerline{\resizebox{.60\textwidth}{!}{\includegraphics{img/logo_urjc}}}
\end{figure}

\begin{center}
  {\Large {\bf MÁSTER UNIVERSITARIO EN SISTEMAS TELEMÁTICOS E INFORMÁTICOS}}
  \vspace{5mm}
 
  {\large {Escuela Técnica Superior de Ingeniería Telecomunicación}}
  \vspace{5mm}

  {\large {Curso académico 2013/2014}}

  \vspace{1cm}

  {\large {\bf Trabajo Fin de Máster}}

  \vspace{2cm}

  {\Large {Estimación multiobjeto en entornos dinámicos\\[1cm] }}

  \vspace{5cm}
  {\bf Autor}: Gonzalo José Abella Dago\\
  {\bf Tutor}: Francisco Martín Rico 
\end{center}

\clearpage
\newpage{\pagestyle{empty}\cleardoublepage}
\thispagestyle{empty}

\vspace{5cm}
\makebox[15cm][r]{
  \begin{tabular}{ll}
  \emph{A mi familia y Emma}
  \end{tabular} 
}

\clearpage
\newpage{\pagestyle{empty}\cleardoublepage}
\thispagestyle{empty}

\vspace{5cm}
\textbf{\huge{Agradecimientos}}\\

Lo cierto es que tengo que agradecer a mucha gente la oportunidad de haber hecho este proyecto. En primer lugar agradecer a Paco, el tutor, todo su esfuerzo y dedicación. He tenido la suerte de trabajar un par de años bajo su tutela y hacer el PFC y el TFM. En todo este tiempo he podido aprender muchísimo de robótica y de muchas otras cosas. ¡Así da gusto trabajar! \\

Como siempre agradecer infinito a la familia, siempre ayudando, siempre apoyando, siempre ahí, a las duras y a las maduras. \\

A la gente del Grupo del Robótica de la URJC, compañía indispensable en el despacho. Al final cada uno acaba por un camino distinto, pero estoy seguro que en algún momento en el futuro nos volveremos a cruzar. \\

Y por último la más importante, Emma. Aguantándome y animándome a diario. Estoy feliz de tenerte. \\

\clearpage
\newpage{\pagestyle{empty}\cleardoublepage}

%%%%%%%%%% RESUMEN %%%%%%%%
\newpage{\pagestyle{empty}\cleardoublepage} 
\vspace{5 cm}
\textbf{\Huge{Resumen}}
\vspace{1.5 cm}

Los robots móviles perciben, miden y \textit{entienden} el entorno que les rodea gracias a sus sensores - láser, ultrasonidos, \textit{bumpers} o cámaras. Esta tarea, que para una persona es básica, para un robot es extremadamente difícil. Uno de los sensores más populares y que más información pueden proporcionar al robot son las imágenes proporcionadas por las cámaras. Analizar estas imágenes es costoso y, normalmente las observaciones no son muy precisas. Por esta razón, es imprescindible filtrar las observaciones para mantener estimaciones fiables y robustas de los estímulos visuales releventas para las tareas que un robot debe llevar a cabo. \\

En este proyecto se ha desarrollado un algoritmo para mantener una estimación de múltiples objetos. Para este propósito se ha empleado una colección dinámica de Filtros Extendidos de Kalman. El estado de estos filtros está compuesto por la posición y velocidad del objeto. La dinámica de los objetos se modela mediante una serie de parámetros simples que hacen que este algoritmo pueda aplicarse a un gran rango de problemas. \\

Como banco de pruebas se ha utilizado en entorno del fútbol de robots. Esta competición es ideal para poner a prueba algoritmos experimentales para, de una forma visual y sencilla, medir su rendimiento y compararlos con el estado del arte actual. \\

El proyecto se ha realizado dentro del Grupo de Robótica de la Universidad Rey Juan Carlos y la solución se ha implementado dentro del código del equipo de la RoboCup SPL SpiTeam como arquitectura base. Se ha desarrollado una implementación para reemplazar el actual algoritmo de seguimiento de la pelota y las porterías.


                                %\pagenumbering{roman}

                                %\setcounter{page}{1}
\frontmatter
\tableofcontents

\listoffigures
                                %\setcounter{page}{1}
                                %\pagenumbering{arabic}
%\listoftables

\mainmatter

% Init counter page
\setcounter{page}{1}

% Capitulo 1
\chapter{Introducción}
\label{cap:introduccion}

\section{Robótica móvil}
\label{cap:roboticamovil}
Los robots móviles son máquinas con la capacidad de desplazarse por un entorno. Para ello hacen uso de un sistema de automoción, ya sean ruedas, patas u orugas. 
La robótica móvil está sufriendo un gran crecimiento debido al abaratamiento del hardware y las grandes oportunidades que nos ofrecen, tanto en materia de educación como en materia industrial. 
Cuando nos encontramos con un problema en el que un robot móvil puede ser la mejor solución tendremos que tener en cuenta los siguientes problemas a solucionar:
\begin{itemize}
\item \textbf{Percepción} Para resolver este problema equiparemos a nuestro robot de sensores, como puede ser una cámara, un láser, sensores de odometría o bumpers para obtener información acerca de qué hay en el escenario en el que se encuentra el robot.
\item \textbf{Localización} Una vez resuelto el problema de conocer que hay cerca del robot, necesitamos conocer la posición del robot dentro del escenario. Esto se puede resolver con algoritmos como MonteCarlo, con cámaras cenitales, triangulación con balizas, etc. Para resolver la localización también es común el uso de mapas. 
\item \textbf{Navegación} La característica principal de los robots móviles es que tienen la capacidad de desplazarse, pero la navegación no solo se cumple cuando el robot se mueve, el robot debe moverse a un punto determinado, de una forma segura, es decir sin chocar con los objetos del escenario, y de la forma más eficaz posible. Dentro de la navegación se pueden distinguir 2 partes, la navegación local que se ocupa del movimiento del robot y de evitar obstáculos y la navegación global qué establece rutas.
\item \textbf{Inteligencia} El siguiente problema se encuentra en un nivel de abstracción más alto. El problema de la inteligencia se refiere a qué tiene que hacer el robot, qué finalidad debe cumplir.
\item \textbf{Autonomía} Habrá momentos en los que el robot deberá tomar algunas decisiones, en referencia a su estado en el escenario, por ejemplo si se acerca mucho a una pared, o si llega a su destino correctamente.
\item \textbf{Interacción con los humanos} Los robots suelen crearse para facilitarnos una tarea o asistirnos en nuestro día a día, por esto los robots deben contar con una interfaz hombre-maquina con la que comunicarnos cosas o con la que nosotros podamos ayudar al robot para así mejorar sus acciones.
\end{itemize}

\section{Mapeado}
\label{cap:mapeado}
Crear un mapa de un escenario puede sernos realmente útil ya que servirán al robot como fuente de información para resolver problemas de Localización y de Navegación. Estos mapas pueden ser creados por un humano, midiendo las paredes y los objetos de un escenario para más tarde transformarlo en una imagen que el robot pueda leer, o puede ser creado directamente con un robot móvil. Esta segunda opción nos permite realizar mapas de zonas que pueden resultar inaccesibles para el ser humano, como puede ser una zona radioactiva u otro planeta del sistema solar.

\begin{figure}[hbtp]
  \begin{center}
    \subfigure[Mapa láser]{\includegraphics[width=6cm,height=5cm]{img/cap1/mapa1}}
    \subfigure[Mapa RGBD]{\includegraphics[width=6cm,height=5cm]{img/cap1/mapa2}}
  \end{center}
  \caption{Ejemplos de mapas}
  \label{fig:maps-ej}
\end{figure}
Para llevar a cabo el mapeo de un escenario con un robot móvil deberemos equipar a nuestro robot con sensores capaces de medir distancias, tales como una cámara RGBD o un láser. La cámara RGBD nos proporciona una nube de puntos en 3D, por lo que nos proporciona una información muy rica del entorno pero el tratamiento de esta información resulta muy costoso. El láser nos proporciona un array de distancias en 270º, estas distancias suelen ser muy precisas y su tratamiento es muy liviano, aunque solo contamos con el plano en el que se ubica el láser. 




\section{RoboCup@Home}
\label{cap:robocup}

\section{Estructura de la memoria}
\label{cap:estructuradelamemoria}




% Capitulo 2
\chapter{Objetivos y Metodología}
\label{cap:objetivos}
Tras haber presentado el contexto en el que se desarrollará este proyecto, en este capítulo se fijarán los objetivos y se expondrán los requisitos que presenta la solución. En primer lugar se describirá el problema y se expondrá la importancia de tener una memoria de los objetos que encontramos en nuestro camino y que inicialmente no estaban en el mapa, después se fijarán los requisitos y los objetivos del proyecto y por último se expondrá la metodología de trabajo seguida.

\section{Descripción del problema y requisitos}
\label{sec:descripciondelproblema}

La navegación en entornos dinámicos, como puede ser una casa o una institución pública, supone un gran reto que superar ya que las personas o las mascotas pueden acercarse o cruzarse delante del robot o podríamos encontrarnos objetos del mobiliario que han sido movidos de su posición original y se encuentran en nuestro camino. En este escenario el robot no debería nunca ni chocar ni perderse en el entorno y debe llegar al destino impuesto por la mejor ruta disponible.

Por ejemplo, si nuestro robot está yendo desde el salón a la cocina de nuestra casa pero en su camino habitual y más optimo se encuentra un mueble, el robot debe darse cuenta rápidamente, esquivarlo, proseguir con su camino y ademas recordarlo para que cuando volvamos al salón de vuelta podamos esquivarlo más fácilmente.Por otro lado en nuestra casa también habrá personas, estas personas se moverán casi constantemente por la estancia por lo que no será del todo correcto tenerlas en cuenta a la hora de planificar nuestra ruta para navegar de un sitio a otro de la casa pero si que será muy importante no chocar con ellas.

Para analizar el entorno del robot usaremos el sensor láser, este sensor destaca por su alta precisión y su corto tiempo de procesado.

El sistema debe cumplir los siguientes requisitos:
\begin{enumerate}
\item \textbf{Robot móvil.} El robot en el que se implemente el algoritmo debe ser un robot móvil.
\item \textbf{Sensor láser.} El robot debe contar con un sensor láser para que los algoritmos de navegación y mapping funcionen correctamente.
\item \textbf{Versátil.} El sistema debe adaptarse a cualquier escenario doméstico sin importar que este sufra cambios.
\item \textbf{Independiente de la plataforma.} Resultará necesario que el algoritmo pueda adaptarse a cambiar de plataforma ya que el robot de nuestras pruebas y el robot de la competición serán distintos.
\end{enumerate}

\section{Objetivo del proyecto}
\label{sec:objetivos}

Se quiere diseñar un algoritmo genere un mapa en tiempo real, el cual será usado por el nodo de navegación de ROS para navegar por un entorno doméstico indicando una posición x,y en el mapa. Este mapa se construirá a partir de la mezcla de 3 mapas, mapa estático, mapa de largo plazo y mapa de corto plazo.
También se creará un sistema de navegación de alto nivel por el que se podrá elegir la estancia del hogar al que queramos que el robot viaje.Este sistema hará uso del nodo de navegación de ROS y del algoritmo de mapping propuesto.
En primera instancia el algoritmo se validará haciendo uso de un simulador en el que se representa una casa con varios tipos de muebles, ya que resulta más fácil de depurar un algoritmo en un entorno virtual, que en un entorno real. Posteriormente el algoritmo se probará en distintas recreaciones de escenarios reales ,y se harán las modificaciones oportunas para adaptarlo al entorno real, y por ultimo se llevará a la competición.

Para simplificar la resolución del problema se ha dividido el proyecto en varios subobjetivos:

\begin{enumerate}
\item Se usarán las herramientas por defecto que nos ofrece ROS para crear el mapa de corto plazo en referencia a las mediciones tomadas por el láser. En un primer paso solo añadiremos los diferentes objetos que percibamos.
\item Se ampliará el algoritmo anterior para poder añadir y eliminar objetos que aparezcan o desparezcan del entorno. 
\item Se desarrollará un algoritmo para mezclar los mapas entre sí y así generar tanto el mapa de largo plazo como el mapa que usaremos para la navegación.
\item Se creará un servidor de mapas dinámicos que se inicializará con los mapas estático y de largo plazo y que generará el mapa final mezclando los mapas de largo plazo y de corto plazo.
\item Se usará el mapa final como parámetro del paquete de navegación de ROS, \textit{move\_base}, y del paquete de localización de ROS \textit{amcl}.
\item Se generará y usará un mapa semántico, en el que cada nivel de gris se asocie con una etiqueta que represente una estancia de la casa, para después ordenar al robot que navegue a dicha estancia.
\end{enumerate}

\section{Metodología de desarrollo}
\label{sec:metodologiadedesarrollo}

En el desarrollo del sistema descrito, el modelo de ciclo de vida utilizado ha sido el modelo en espiral basado en prototipos. Este modelo permite desarrollar el proyecto de forma incremental, aumentando la complejidad progresivamente y haciendo posible la generación de prototipos funcionales. Este planteamiento permite obtener productos parciales al final de cada ciclo que pueden ser evaluados, ya sea total o parcialmente. Esto facilita la adaptación a los cambios en los requisitos, circunstancia muy común en los proyectos de investigación.

\begin{figure} [hbtp]
  \begin{center}
    \includegraphics[width=16cm]{img/cap2/modelo_espiral}
  \end{center}
  \caption{Modelo en espiral.}
  \label{fig:modelo_espiral}
\end{figure}

Este modelo está dividido en ciclos. Cada ciclo representa una fase del proyecto y está dividido, a su vez, en 4 partes. Cada una de las partes tiene un objetivo distinto:

\begin{itemize}
\item \textbf{Determinar objetivos.} Se establecen las necesidades que debe cumplir el sistema en cada iteración teniendo en cuenta los objetivos finales. Según avanzan las iteraciones aumenta el coste del ciclo y su complejidad.
\item \textbf{Evaluar alternativas.} Se determinan diferentes formas de alcanzar los objetivos que se han establecido en la fase anterior. Se aborda el problema desde distintos puntos de vista, como, por ejemplo, el rendimiento del algoritmo en tiempo y espacio. Además, se consideran explícitamente los riesgos, intentando mitigarlos al máximo.
\item \textbf{Desarrollar y verificar.} Se desarrolla el producto siguiendo la mejor alternativa para poder alcanzar los objetivos del ciclo. Una vez diseñado e implementado el producto, se realizan las pruebas necesarias para comprobar su funcionamiento.
\item \textbf{Planificar.} Teniendo en cuenta los resultados de las pruebas realizadas, se planifica la siguiente iteración, se revisan los errores cometidos y se comienza un nuevo ciclo de la espiral.
\end{itemize}

\section{Plan de trabajo}
\label{sec:plandetrabajo}

Para poder abordar el problema se han marcado una serie de subobjetivos ha completar. Dichos hitos son los siguientes:

\begin{enumerate}
\item Estudio y comprensión de la composición de un mapa y como construirlo. Nos apoyaremos en las herramienta ofrecidas por ROS y que resultaran básicas para este fin, dicha herramientas son \textit{TF} \footnote{http://wiki.ros.org/tf} y \textit{Costmap}\footnotemark .
\footnotetext{http://wiki.ros.org/costmap\_2D}
\item Primer subobjetivo. Una vez conocido como funcionan los \textit{costmap} procederemos a crear un pequeño nodo en el que se cree un mapa con las observaciones instantáneas que percibimos con el láser.
\item Segundo subobjetivo. Extender el algoritmo anterior para poder añadir y eliminar objetos según entren o salgan de la escena.
\item Tercer subobjetivo. Modificar el paquete \textit{map\_server} para que acepte varios mapas como entrada y estudiar la manera de mezclar los mapas entre sí.
\item Fase de pruebas. Se le pasará al paquete de navegación de ROS, \textit{move\_base} ,el mapa resultante y se harán pruebas de navegación en el simulador y en el robot real.
\item Cuarto subobjetivo. Crear el mapa semántico, especificando las etiquetas naturales que tendrá una casa, salón, cocina, habitación... y usarlo para poder navegar a la estancia que le indiquemos.
\end{enumerate}


% Capitulo 3
%\chapter{Entorno y plataforma de desarrollo}
\label{cap:entorno}

En este capítulo se describen los elementos que componen el sistema. El desarrollo del algoritmo no se limita únicamente a una implementación independiente a la que se somete a pruebas sintéticas para ver que se comporta según lo esperado, sino que se ha integrado dentro de un sistema robótico completo y funcional. Para poder hacer esto, es necesario adquirir una serie de conocimientos de todo el entorno de este sistema. Si bien no es necesario que este conocimiento sea exhaustivo y se tengan que conocer absolutamente todos los detalles, sí que hay que tener una idea muy clara de la arquitectura y su funcionamiento. \\

A continuación se explican los distintos componentes que se han utilizado para elaborar y probar el algoritmo. Se mostrará la plataforma robótica con la que se ha trabajado, se explicará la arquitectura del sistema robótico, las herramientas utilizadas y el algoritmo base sobre el que se sustenta la solución: el Filtro Extendido de Kalman. Para comenzar, en la figura \ref{fig:sistema}, se muestra un esquema general del funcionamiento del sistema y la conexión entre los distintos componentes. \\

\begin{figure} [h]
  \begin{center}
    \includegraphics[width=15cm]{img/cap3/sistema}
  \end{center}
  \caption{Esquema general del funcionamiento del sistema}
  \label{fig:sistema}
\end{figure}

El robot capta las imágenes mediante una de las cámaras situadas en la cabeza. Normalmente se utiliza la cámara situada en la boca, porque está orientada hacia el suelo y facilita ver la pelota. La imagen es procesada por los detectores, que extraen los objetos interesantes que se encuentran en ella. En este caso serían los postes y la pelota. A continuación se procesan y filtran las observaciones para minimizar ruido, detectar posibles falsos positivos y fusionar con el resto de observaciones anteriores. El algoritmo desarrollado en este proyecto se encarga precisamente de esto. Su misión es obtener datos más robustos y fiables de una forma eficiente a partir de las observaciones de los detectores.\\

\section{Robot Nao}
\label{sec:robotnao}

 Nao es un robot humanoide de 58 cm de altura, figura \ref{fig:gradoslibertad}. Lo desarrolla la empresa francesa \textit{Aldebaran Robotics}\footnote{http://www.aldebaran-robotics.com/}. El proyecto de desarrollo del Nao nace en 2004. Al poco tiempo, en 2007, reemplaza al robot Aibo\footnote{http://es.wikipedia.org/wiki/Aibo}, creado por Sony, como el robot usado en la competición de la \textit{RoboCup Standard Platform League (SPL)}. En 2011 Aldebaran anuncia que van a lanzar el código fuente del controlador del Nao como software libre. Por el momento este robot es utilizado mayoritariamente en el ámbito académico, pero fuera de éste tiene muchos usos aún por descubrir. Las principales características del robot Nao son:

\begin{packed_item}
\item Gran cantidad de movimientos. El \textit{Nao RoboCup Edition}, que es la versión utilizada para la RoboCup y la que se utiliza en este proyecto, cuenta con 21 grados de libertad.
\item Detectores de presión en pies y manos, conocidos como FSR, \textit{Force Sensitive Resistors}. Este dispositivo tiene la capacidad de disminuir su resistencia cuando aumenta la fuerza aplicada sobre él. Suele utilizarse para el control de dispositivos electrónicos con el tacto.
\item Dos cámaras situadas en la cabeza con distintas zonas de visión. Una de ellas está situada en la frente del robot y apunta hacia el frente. La otra cámara esta situada en la boca del robot y tiene cierta inclinación hacia abajo. Los ángulos de visión de las cámaras no se solapan, en la figura \ref{fig:zonasvision} se puede ver hacia donde apunta cada cámara. Ambas tienen una resolución de 640x480 píxeles.
\item Cuatro sensores de ultrasonido colocados en el pecho del robot.
\item Un sensor inercial que mide la aceleración y la velocidad angular. Este sensor es muy útil en caso de caídas para detectarlas y actuar en consecuencia.
\item Interfaces de red Ethernet y WiFi que aportan conectividad al robot.
\item LEDs con distintos colores repartidos por el cuerpo del robot. Tiene uno en el botón de encendido y apagado del pecho del robot, un par en los ojos y varios en las orejas y pies.
\item Cuatro micrófonos colocados en la parte frontal, la parte posterior, al lado derecho y al lado izquierdo del robot.
\item Dos altavoces Hi-Fi en estéreo para reproducir sonidos colocados en la cabeza del robot.
\end{packed_item}

\begin{figure}[hbtp]
  \centering
  \subfloat[Sensores y actuadores del robot.]{
    \label{fig:gradoslibertad}
    \includegraphics[width=9cm]{img/cap3/gradoslibertad}
  }
  \subfloat[Zonas de visión de cada una de las cámaras.]{
    \label{fig:zonasvision}
    \includegraphics[width=6cm]{img/cap3/zonasvision}
  }
  \caption{El robot Nao.}
  \label{fig:robotnao}
\end{figure}

El robot dispone de un procesador x86 AMD Geode 500MHz y usa como sistema operativo Linux. Se alimenta de una batería recargable que le permite funcionar durante aproximadamente 45 minutos o durante quince minutos caminando sin parar. Este tiempo es suficiente para jugar una de las partes de un partido.\\

La apariencia de este robot es agradable y amigable. En los últimos años se está ampliando el ámbito de usos de este robot, que no se limita únicamente al fútbol robótico. Por ejemplo, hasta hace un año el Grupo de Robótica de la Universidad Rey Juan Carlos, estaba utilizando este robot en sesiones de Roboterapia con enfermos de Alzheimer en colaboración con la Fundación CIEN\footnote{http://www.fundacioncien.es}, \cite{Alzheimer2013} y \cite{RobotsEnTerapias2012}.

\section{NaoQi}
\label{sec:naoqi}

\textit{NaoQi} es un Framework, creado por Aldebaran Robotics, que permite desarrollar aplicaciones en C++ y Python en el Nao. Facilita el acceso a los sensores y actuadores del robot. Las aplicaciones creadas pueden ser ejecutadas directamente por el robot o remotamente en un ordenador.\\

Los ejecutables creados con este Framework se llaman \textit{broker}. Los \textit{broker} se ejecutan de forma independiente y se encuentran escuchando en una dirección IP y un puerto, por lo que pueden ser ejecutados en el propio robot, usando un compilador cruzado proporcionado por el Framework, o remotamente desde un ordenador. Un \textit{broker} esta formado por una serie de módulos que ofrecen distintas funcionalidades. Las funciones de estos módulos pueden ser llamadas desde otros módulos o incluso desde otros \textit{broker}.\\

\begin{figure} [h]
  \begin{center}
    \includegraphics[width=9cm]{img/cap3/broker-naoqi}
  \end{center}
  \caption{Arquitectura de NaoQi modulada por medio de \textit{Brokers}.}
  \label{fig:broker-naoqi}
\end{figure}

En la figura \ref{fig:broker-naoqi} se puede ver un esquema de la arquitectura de \textit{NaoQi} modulada por medio de los \textit{broker}. El \textit{broker} más importante es el \textit{MainBroker} porque nos da acceso a los sensores y actuadores del robot. Cuando se desarrolla una aplicación para el robot con este Framework, se puede ejecutar a través de un \textit{broker} propio o como un módulo del \textit{MainBroker}; BICA se ejecuta de esta última manera. \textit{NaoQi} se utiliza para acceder a los sensores y actuadores de una manera más sencilla.\\

Los \textit{brokers} se organizan internamente en módulos. Cada uno de estos módulos aporta una funcionalidad concreta o permite acceder a sensores o actuadores del robot. Por ejemplo, \texttt{ALMotion} proporciona métodos que facilitan hacer que el robot se mueva; \texttt{ALAudioDevice} contiene otros módulos de \textit{NaoQi} que nos dan acceso a las entradas y salidas de audio; \texttt{ALVideoDevice} se encarga de proporcionar imágenes de las cámaras, y \texttt{ALSensors} es responsable de lanzar los eventos correspondientes cuando se pulsa un botón o se tocan las zonas táctiles de la cabeza o las manos.

\section{BICA}
\label{sec:bica}

BICA\footnote{http://www.robotica-urjc.es/index.php/Robocup}, \textit{Behavior-based Iterative Component Architecture}, es un software desarrollado en el grupo de Robótica de la Universidad Rey Juan Carlos, \cite{BICA2010} y \cite{BICA2013}. Se trata de una plataforma de desarrollo de software para el robot Nao. En la figura \ref{fig:bloques-bica} se puede ver la división en capas de la arquitectura de BICA.

\begin{figure} [hbtp]
  \begin{center}
    \includegraphics[width=7cm]{img/cap3/bloques-bica}
  \end{center}
  \caption{Arquitectura de BICA dividido en capas.}
  \label{fig:bloques-bica}
\end{figure}\

La unidad básica en la arquitectura de BICA es el \textit{componente}, representado en la figura \ref{fig:componente-bica}. La funcionalidad de los \textit{componentes} puede ser implementada mediante una máquina de estados o pueden ser controladores reactivos, es decir, que ejecutan al momento la acción requerida. Los \textit{componentes} pueden activarse o desactivarse. Un \textit{componente} activo ejecuta una tarea determinada de manera iterativa y con una frecuencia previamente fijada. Los \textit{componentes} que están activos iteran y consumen recursos. Los \textit{componentes} tienen una serie de métodos para modularlos y devuelven los resultados obtenidos.\\

\begin{figure} [hbtp]
  \begin{center}
    \includegraphics[width=9cm]{img/cap3/componente-bica}
  \end{center}
  \caption{Esquema de las entradas y salidas de un componente de BICA.}
  \label{fig:componente-bica}
\end{figure}

Para realizar tareas más complejas los \textit{componentes} pueden comunicarse entre sí. La arquitectura de BICA tiene una estructura jerárquica, tal y como se muestra en la figura \ref{fig:componentes-bica}. Cuando se activa el \textit{componente A}, éste activa los \textit{componentes D} y \textit{E}. A su vez, el componente \textit{D} activa los \textit{componentes G} y \textit{H}, y el \textit{componente E} activa los \textit{componentes H}, \textit{I} y \textit{J}. Un \textit{componente} puede ser activado por varios \textit{componentes}. Aunque sea llamado repetidas veces, éste se ejecutará a la frecuencia mínima que tenga configurada para evitar ciclos innecesarios y se ahorren recursos. Los \textit{componentes} de ''bajo nivel'', como los \textit{componentes G}, \textit{H}, \textit{I} y \textit{J}, se comunican directamente con el robot u obtienen la información mediante llamadas de \textit{NaoQi}.\\

\begin{figure} [hbtp]
  \begin{center}
    \includegraphics[width=9cm]{img/cap3/componentes-bica}
  \end{center}
  \caption{Jerarquía de componentes de BICA.}
  \label{fig:componentes-bica}
\end{figure}

A continuación se muestra un ejemplo en pseudocódigo muy básico del funcionamiento de los \textit{componentes}. Los \textit{componentes} tienen un par de métodos indispensables: \texttt{init()} y \texttt{step()}. Desde el método \texttt{init()} se inicializan los recursos necesarios para ejecutar el \textit{componente}. El método \texttt{step()} es el que activa el \textit{componente} y contiene toda su funcionalidad. Como el \textit{componente} se activa solamente cuando se llama a este método, no es necesario ningún otro método para detener su ejecución. Todos los \textit{componentes} disponen de un método privado llamado \texttt{isTime2Run()} que modula la frecuencia a la que se ejecuta el \textit{componente}. Si aún no tiene que ejecutarse el \textit{componente}, el método devuelve \texttt{false} y no se ejecuta ninguna instrucción del \textit{componente}.

\begin{lstlisting}[style=C]
void step() {
  // Ejecucion en cascada de los componentes de los que se
  // obtienen informacion.
  cp1->step();
  cp2->step();

  if (isTime2Run()) {
    // Recoger datos de los componentes perceptivos.

    // Iteracion genuina

    // Poner a disposicion del resto de componentes los datos
    // calculados y/o se modulan los componentes actuadores.
  }

  // Ejecucion en cascada de los componentes que se han
  // modulado o requieran ser ejecutados por su actuacion.
  ca1->step();
  ca2->step();
}
\end{lstlisting}

Al inicio del método \texttt{step()} se llama a los \textit{componentes} perceptivos, que son aquellos que devuelven datos, de los que depende el \textit{componente}. Si es el momento de que se ejecute el \textit{componente}, el método \texttt{isTime2Run()} devuelve \textit{true} y se ejecutan las instrucciones dentro de la estructura \textit{if}. Básicamente, lo que se hace en esta estructura es procesar los datos devueltos por los \textit{componentes} perceptivos y generar nuevos datos. En caso de que el \textit{componente} genere una respuesta, se modulan los \textit{componentes} actuadores, aquellos que generan una respuesta en el robot o procesan los datos que acabamos de crear. Por último, se llama al \texttt{step()} de éstos para que efectúen las acciones requeridas que acabamos de modular.\\

BICA está desarrollado en C++ y consiste en una arquitectura iterativa, formada por \textit{componentes} y basada en comportamientos. Proporciona un entorno de programación mono-hilo. Gracias a que se ejecuta en un sólo hilo nos evita las condiciones de carrera, un error muy frecuente y difícil de detectar en la programación concurrente.\\

Se ha escogido esta plataforma software por ser robusta y estar bastante probada. Ya ha sido utilizada en varios proyectos y es la plataforma utilizada en la RoboCup por el equipo de la Universidad. Además, dispone de varios \textit{componentes} que se pueden reutilizar. \\

Las comunicaciones con agentes externos se realizan a través del motor de comunicaciones de Internet, ICE, \textit{Internet Communications Engine}. Se trata de un \textit{middleware} de computación distribuida, orientado a objetos, multiplataforma y es desarrollado por la empresa \textit{ZeroC}\footnote{http://www.zeroc.com/}. Este \textit{middleware} proporciona una solución simple en el ámbito de las comunicaciones entre aplicaciones distribuidas en distintos servidores.\\

ICE dispone de una versión para sistemas embebidos, \textit{Ice-E}. Esta versión es un motor de comunicaciones más compacto diseñado para ejecutarse en entornos de recursos limitados, como teléfonos inteligentes o PDAs, por poner un par de ejemplos. Esta es la versión utilizada en el robot, ya que los recursos son bastante limitados y se necesita que el software corra lo más rápidamente posible. Gracias a ICE, BICA puede comunicarse con \textit{JManager} o con otros robots que usen BICA.

\subsection{Sistema de visión}
\label{subsec:sistemadevision}

La carga de trabajo del sistema de visión está distribuida en varios componentes. Básicamente el objetivo de este sistema es el de obtener, a partir de una imagen, los objetos interesantes que se encuentren en ella. Esta tarea de divide en varias subtareas más simples y abordables. El diagrama \ref{fig:visionsystem} muestra una secuencia de estos pasos.

\begin{figure} [h]
  \begin{center}
    \includegraphics[width=15.5cm]{img/cap3/vision_system_real_images}
  \end{center}
  \caption{Esquema de funcionamiento del sistema de visión}
  \label{fig:visionsystem}
\end{figure}

En el momento en el que se obtiene una imagen se calculan lo que en el diagrama se ha etiquetado como \textit{kinematics}, en referencia al cálculo de los parámetros cinemáticos del robot. Se calcula la posición relativa de la cámara respecto del centro del robot. La cámara cuenta con tres grados de libertad: \textit{pan}, \textit{tilt}, \textit{roll}. Cada uno de estas medidas se corresponde con uno de los ejes del espacio tridimensional, en la figura \ref{fig:visionsystem} se puede ver a cuál corresponde cada uno de ellos. A partir de estos valores se averigua la matriz de traslación y rotación de la cámara, lo que nos permite transformar píxeles de la imagen 2D a puntos del entorno en 3D y viceversa. \\

Al mismo tiempo se procesa la imagen para extraer las características más interesantes de esta. Analizar imágenes es un proceso muy costoso que se repite muchas veces por segundo. Es primordial utilizar algoritmos rápidos y eficaces para que el robot pueda responder rápidamente a los estímulos. El algoritmo utilizado en BICA se conoce coloquialmente como \textit{rastrillado} de la imagen. En el diagrama es el paso etiquetado como segmentación. En vez de analizar completamente la imagen, se analizan una serie de columnas de esta. La resolución es ajustable. Cuanta menor es la resolución mayor es la velocidad de ejecución del algoritmo a costa de perder precisión. El algoritmo actual consume aproximadamente unos 8 milisegundos en analizar la imagen. \\

Para simplificar las tareas de análisis de las imágenes, el entorno de la RoboCup presenta una serie de colores bien definidos: las porterías son amarillas o azules, la pelota es naranja, el campo es verde, etc. Esta configuración permite filtrar la imagen por colores. Esta operación, comparada con otras técnicas de análisis de la imagen más avanzadas, es bastante liviana computacionalmente. Esto no quiere decir que sea \textit{pan comido}. Los algoritmos de visión tienen que ser igualmente robustos, ya que cualquier cambio en la intensidad de la luz puede afectar el rendimiento y resultado de estos. \\

Una vez segmentada la imagen, hay que reconstruirla, paso etiquetado como reconstrucción. Para ello, se analizan todos los segmentos de la imagen y se conectan los que tengan una relación de proximidad, es decir, se conectan los segmentos que tengan el mismo color y se encuentren a la izquierda, derecha, arriba o abajo el uno del otro. Al unir varios segmentos se regenera la imagen con masas amorfas de colores o \textit{blobs}. En un entorno como la RoboCup se puede sacar mucha información solamente a partir de estos \textit{blobs}. Según la resolución escogida, este conjunto de manchas se asemejará en mayor o menor medida a la realidad. \\

La detección de objetos en la imagen se hace a partir de la imagen reconstruida. Se comprueba que el color del objeto sea el correcto y se valida la observación teniendo en cuenta el tamaño y la forma real de dicho objeto. Aparte de estas validaciones básicas, hay que tener cuidado con objetos que se encuentren fuera del terreno de juego, pero que puedan asemejarse a cualquier elemento utilizado en el partido de fútbol. Véase el ejemplo de la \textit{niña baliza}, figura \ref{fig:ninabaliza}. En ediciones anteriores de la RoboCup se utilizaban una serie de balizas para simplificar la autolocalización del robot dentro del campo. En un campeonato apareció una niña vestida con los mismos colores que la baliza. Un método para apaliar estos falsos positivos es utilizar una frontera visual que limite el espacio donde realizar la búsqueda de un objeto. La frontera visual dibujada en el paso de reconstrucción de la figura \ref{fig:visionsystem} es un ejemplo de ello. Sólo es aplicable a la detección de la pelota, no de las porterías. El espacio de búsqueda se limita, como mucho, a un radio de 6 metros alrededor del robot. Con esto se asegura que cualquier objeto fuera de este radio está fuera del terreno de juego y se ignora. \\

\begin{figure} [h]
  \begin{center}
    \includegraphics[]{img/cap3/nina_baliza}
  \end{center}
  \caption{La niña baliza}
  \label{fig:ninabaliza}
\end{figure}

Para calcular la posición de los objetos detectados en la imagen se utiliza la hipótesis suelo, figura \ref{fig:hipotesis-suelo}. De cada uno de los objetos se obtiene el píxel en el que intersecta con el suelo. A partir de este píxel -del cual se conoce la altura que es 0- la matriz de rotación y traslación de la cámara con respecto del robot calculada en el primer paso y una serie de cálculos geométricos, se calcula el punto 3D correspondiente con dicho píxel. Esta posición lleva un ruido asociado a ella. Todo junto es lo que se considera una observación, que es el dato de entrada del algoritmo desarrollado en este proyecto. \\

\begin{figure} [h]
  \begin{center}
    \includegraphics[width=9cm]{img/cap3/hipotesis_suelo}
  \end{center}
  \caption{Hipótesis suelo}
  \label{fig:hipotesis-suelo}
\end{figure}

\section{JManager}
\label{sec:jmanager}

JManager es una aplicación de escritorio desarrollada en Java que contiene herramientas para la monitorización, configuración y depuración de los algoritmos del robot. La figura \ref{fig:jmanagerscreenshot} está compuesta por un par de capturas de pantalla de esta aplicación. La aplicación JManager está organizada en varias pestañas, donde cada una proporciona una funcionalidad distinta. Una de estas funcionalidades es la activación y desactivación de los componentes disponibles en el robot, figura \ref{fig:jmanagerscreenshot02}. De igual manera se puede activar el modo de depuración y la interfaz gráfica propia de cada componente para modularlo y depurarlo. \\

\begin{figure}[h]
  \centering
  \subfloat[Administración de la conexión con el robot]{
    \label{fig:jmanagerscreenshot01}
    \includegraphics[width=6.5cm]{img/cap3/jmanager_screenshot01}
  }
  \subfloat[Administración de componentes]{
    \label{fig:jmanagerscreenshot02}
    \includegraphics[width=7cm]{img/cap3/jmanager_screenshot02}
  }
  \caption{Capturas de pantalla de JManager}
  \label{fig:jmanagerscreenshot}
\end{figure}

Las conexiones con el robot se realizan a través de la red con ayuda de ICE. La figura \ref{fig:jmanager} es un esquema que representa la conexión entre JManager con la interfaz de ICE de cada componente. Todos los componentes que requieren de una conexión con el exterior, ya sea sólo para depuración o para comunicarse con otros agentes, implementan una de estas interfaces. En este caso la conexión se realiza desde el JManager, pero en caso de querer extender la funcionalidad con nuevas herramientas, sólo hay que implementar dichas interfaces. No es necesario tocar el código del robot. Esta capa de abstracción aporta una mayor independencia entre el código del robot y las herramientas de configuración y depuración.

\begin{figure} [h]
  \begin{center}
    \includegraphics[width=12cm]{img/cap3/jmanager}
  \end{center}
  \caption{Conexión entre JManager y BICA.}
  \label{fig:jmanager}
\end{figure}

\subsection{Depurador de la memoria visual}
\label{subsec:depuradormemoriavisual}

Uno de los módulos que tiene JManager y que merece una mención especial es el depurador de la memoria visual, figura \ref{fig:depuradormemoriavisual}. El depurador consta de un espacio en la parte izquierda donde se pueden pintar distintas figuras geométricas representando los objetos que percibe el robot. La parte de la derecha se muestra información más detallada de algunos de los objetos más importantes del campo en una tabla, aunque en esta captura de pantalla no se está utilizando. \\

\begin{figure} [h]
  \begin{center}
    \includegraphics[width=15cm]{img/cap3/depurador_memoria_visual}
  \end{center}
  \caption{Captura de pantalla del depurador de memoria visual}
  \label{fig:depuradormemoriavisual}
\end{figure}

Los objetos que se muestran son un par de postes y la pelota. La incertidumbre de estos objetos se representa mediante una elipse. Cuanto menor es el tamaño de la elipse, menor es la incertidumbre. Esta representación visual es muy útil ya que de un solo vistazo se muestra mucha información. Por ejemplo, se puede ver claramente que los postes tienen más incertidumbre que la pelota. También se puede ver que, en general, se tiene mayor incertidumbre en cuanto a la distancia que en cuanto al ángulo con respecto del robot que se encuentran los objetos. Como la incertidumbre de las dos variables es distinta, la figura geométrica es una elipse. Si por el contrario la distancia y al ángulo tuviesen la misma incertidumbre, la figura sería un círculo. \\

El depurador se ejecuta en tiempo real y permite ver en todo momento en qué estado se encuentra la memoria de los objetos del robot: la posición en la que se encuentran y su incertidumbre. Esta herramienta es básica para la depuración y validación de los algoritmos de visión.

\section{Simulador}
\label{sec:simulador}

Los simuladores son herramientas que simulan un entorno real con todas sus propiedades físicas. Es una herramienta indispensable para el desarrollo de algoritmos complejos en robótica, ya que aumenta mucho el ritmo de trabajo. \\

El robot y los objetos simulados se definen mediante una serie de propiedades, como, por ejemplo, la forma, el color, la textura y la masa. También se simulan los sensores del robot, de manera que el robot sólo percibe lo que le llega a través de estos. Es importante que la simulación represente lo más fielmente posible la realidad para que los algoritmos desarrollados funcionen después en un robot real. \\

Para demostrar la importancia de los simuladores en la robótica, en la figura \ref{fig:darpavrc} se puede ver una par de imágenes de una competición que organiza el Departamento de Defensa de EE.UU.\footnote{http://www.darpa.mil/Our\_Work/TTO/Programs/DARPA\_Robotics\_Challenge.aspx} en el que se programa un robot humanoide, Atlas\footnote{http://es.wikipedia.org/wiki/Atlas\_(robot)},  para realizar una serie de tareas complejas tales como desplazarse por terrenos complejos, subirse y bajarse de un buggy, conducir el buggy con volante y pedales o enchufar una manguera a una boca de riego y abrir la válvula, que son solo algunas de ellas. Los ganadores de la competición reciben una importante suma de dinero, un robot Atlas y el derecho a participar en una competición similar a la simulada, pero con los robots reales. Y todo ello habiendo utilizado sólamente un simulador. \\

\begin{figure}[h]
  \centering
  \subfloat[Enchufando una manguera a una boca de riego]{
    \label{fig:atlasmanguera}
    \includegraphics[width=7cm]{img/cap3/atlas_manguera}
  }
  \subfloat[Conduciendo un buggy]{
    \label{fig:atlasdriving}
    \includegraphics[width=7.7cm]{img/cap3/atlas_driving}
  }
  \caption{Robot Atlas en un par de pruebas de la DARPA VRC}
  \label{fig:darpavrc}
\end{figure}

Las pruebas de validación de este proyecto se han realizado en un escenario idéntico a un campo oficial de la RoboCup SPL. El algoritmo que se ejecuta es exactamente el mismo que luego se ejecuta en el robot real. La única diferencia es que la imagen procesada es una imagen del simulador en vez de ser una real. \\

\section{Estimación probabilística: Filtro Extendido de Kalman}
\label{sec:filtroextendidodekalman}

El Filtro de Kalman (KF) es un algoritmo de procesado de datos óptimo y recursivo. Este método fue descrito por Rudolf E. Kalman en 1958 para estimar los estados de un sistema estocástico. Estima el estado no observable de un sistema dinámico dado un conjunto de observaciones que proporcionan información acerca del estado en cada instante. Al ser recursivo no es necesario mantener los datos previos. Esta información se encuentra implícita en el estado, hecho que facilita en gran medida su implementación en sistemas de tiempo real. El algoritmo de Kalman estima los estados de manera óptima minimizando el índice del error cuadrático medio.\\

El KF solo es aplicable en sistemas lineales. Cuando se modela un entorno real los sistemas casi nunca son lineales, son más complejos. Por ejemplo, cuando un objeto rota alrededor del robot con una trayectoria circular, ésta no puede describirse como una transición lineal. El Filtro Extendido de Kalman (EKF) es la versión para sistemas no-lineales del KF. A costa de ampliar la familia de sistemas dinámicos el EKF pierde la optimalidad del algoritmo, aunque puede llegar a serlo si, tanto la transición del estado como la incorporación de las nuevas medidas, son lineales. Además, si la estimación inicial del estado es incorrecta o el proceso está mal modelado, el filtro puede diverger rápidamente. A pesar de estos inconvenientes, el EKF se considera el algoritmo \textit{por defecto} en sistemas de navegación y GPS.\\

El Filtro de Kalman representa la estimación de un estado en un instante de tiempo. En el período de tiempo $t$, esta estimación se representa por la media $s_t$ y la covarianza $P_t$. El cálculo del siguiente estado y la fase de corrección están dirigidas por las funciones no lineales $g$ y $h$, respectivamente:
\begin{equation}
s_t = g(u_t, s_{t-1}) + \epsilon_t
\end{equation}
\begin{equation}
z_t = h(s_t) + \delta_t
\end{equation}
donde $s_t$ y $s_{t-1}$ son vectores de estado, y $u_t$ es el vector de control en el tiempo $t$. Ambos vectores son verticales y tienen la forma
\begin{equation}
x_t = \begin{pmatrix}
      x_{1,t} \\ 
      x_{2,t} \\ 
      \vdots \\ 
      x_{n,t}
      \end{pmatrix}
\text{ y }
u_t = \begin{pmatrix}
      u_{1,t} \\ 
      u_{2,t} \\ 
      \vdots \\ 
      u_{m,t}
      \end{pmatrix}
\end{equation}

La idea que subyace en el EKF es la linealización. Existen muchas técnicas para linealizar funciones no lineales. EKF utiliza el método conocido como \textit{expansión de Taylor}. Este método construye una aproximación lineal de una función $g$ a partir del valor actual de $g$ y su pendiente. La pendiente se calcula con la derivada parcial de la función
\begin{equation}
g'(u_t, s_{t-1}) := \frac{\partial g(u_t, s_{t-1})}{\partial s_{t-1}}
\end{equation}

El estado de una sistema de Kalman normalmente tiene más de una dimensión. En estos casos en vez de la derivada simple se utiliza la matriz \textit{Jacobiana}. Si $f:\mathbb{R}^n \rightarrow \mathbb{R}^m$ es una función que va del espacio euclídeo $n$-dimensional a otro espacio euclídeo $m$-dimensional, las derivadas parciales de estas $m$ funciones se organizan en una matriz de tamaño de $m \times n$. \\
\begin{equation}
  J = 
  \begin{pmatrix}
    \frac{\delta f_1}{\delta x_1} & \cdots & \frac{\delta f_1}{\delta x_n} \\
    \vdots                        & \ddots &                        \vdots \\
    \frac{\delta f_m}{\delta x_1} & \cdots & \frac{\delta f_m}{\delta x_n} \\    
  \end{pmatrix}
\end{equation}

\subsection{El algoritmo EKF}

El algoritmo puede verse en la ecuación \ref{eq:ekf_algorithm}. El EKF representa la estimación en el tiempo $t$ de la media $s_t$ y la covarianza $P_t$. Los datos de entrada del filtro es la estimación en el tiempo $t-1$, que se representan como $s_{t-1}$ y $P_{t-1}$. Para actualizar estos parámetros, son necesarios los parámetros de control $u_t$ y las observaciones $z_t$. El dato de salida es la estimación en el tiempo $t$, que se representa por $s_t$ y $P_t$.
\begin{equation}
\label{eq:ekf_algorithm}
\begin{array}{ll}
1: & \large{\textbf{EKF}}(u_{t-1}, P_{t-1}, u_t, z_t) \\
2: & \bar{s_t} = g(s_{t-1}, u_t, 0) \\
3: & \bar{P_t} = A_t P_{t-1} A^T_t + W_t Q_t W^T_t \\
4: & K_t = \bar{P_t} H^T_t ( H_t \bar{P_t} H^T_t + V_t R_t V^T_t )^{-1} \\
5: & s_t = \bar{s_t} + K_t ( z_t - h( \bar{s_t}, 0 ) ) \\
6: & P_t = ( I - K_t H_t ) \bar{P_t} \\
7: & \mbox{return } s_t, P_t
\end{array}
\end{equation}

En las líneas 2 y 3, la estimación predecida $\bar{s}$ y $\bar{P}$ se calcula justo antes de incorporar la observación $z_t$. Esta estimación se obtiene incorporando los datos proporcionados por el vector de control $u_t$. Entre las líneas 4 y 6 se incorpora la observación $z_t$. La matriz de la \textit{Ganancia de Kalman} $K_t$ dice cuánto se confía en la nueva observación. El valor acotado por la llave se conoce como la \textit{innovación}.
\begin{equation}
4: s_t = \bar{s_t} + K_t \overbrace{( z_t - h( \bar{s_t}, 0 ) )}^\text{innovación}
\end{equation}
Es la diferencia entre la observación y la predicción. Un valor alto de la Ganancia de Kalman significa que se confía más en la observación que en la predicción, por lo que, el nuevo estado será más parecido a la observación que a la predicción. Si la ganancia de Kalman tiene un valor bajo, ocurre lo contrario. No se confía en la nueva observación y el nuevo estado será más cercano a la predicción que a la observación.\\

Las matrices $A$, $W$, $H$ y $V$ son las matrices Jacobianas obtenidas de la linealización del modelo del sistema. $A$ y $W$ son las jacobianas de $g()$ con respecto al estado $s$ y al ruido en la predicción del sistema $w$. $H$ y $V$ son las jacobianas con respecto a $s$ y al ruido de la observación $v$. Estas matrices se calculan a partir del estado actual $s$.

\subsection{Representación gráfica del EKF}
\label{subsec:representaciondelekf}

La figura \ref{fig:kalmangrafico} es una representación de cómo funciona el algoritmo de Kalman. Esta representación se hace sobre un modelo simple compuesto por una sola dimensión, pero permite ver y entender cómo funciona el algoritmo de Kalman muy fácilmente. \\

\begin{figure}
  \centering
  \begin{tabular}{cc}
  \subfloat[estimación inicial]{
    \label{fig:kalman1}
    \includegraphics[width=0.45\textwidth]{img/cap3/kalman1}
  } &
  \subfloat[una observación (en negrita) con una incertidumbre asociada]{
    \label{fig:kalman2}
    \includegraphics[width=0.45\textwidth]{img/cap3/kalman2}
  } \\
  \subfloat[estimación después de incorporar la observación a la estimación usando el algoritmo de un EKF]{
    \label{fig:kalman3}
    \includegraphics[width=0.45\textwidth]{img/cap3/kalman3}
  } &
  \subfloat[estimación después de un desplazamiento hacia la derecha, que introduce incertidumbre]{
    \label{fig:kalman4}
    \includegraphics[width=0.45\textwidth]{img/cap3/kalman4}
  } \\
  \subfloat[una nueva observación con una incertidumbre asociada]{
    \label{fig:kalman5}
    \includegraphics[width=0.45\textwidth]{img/cap3/kalman5}
  } &
  \subfloat[la incertidumbre asociada]{
    \label{fig:kalman6}
    \includegraphics[width=0.45\textwidth]{img/cap3/kalman6}
  }
  \end{tabular}
  \caption{Ejemplo gráfico de un EKF}
  \label{fig:kalmangrafico}
\end{figure}

Se supone el caso de un robot que se mueve horizontalmente a lo largo de un raíl o sobre un eje. La estimación de la posición del robot se representa mediante una distribución normal. La estimación inicial del robot de donde está localizado se muestra en la figura \ref{fig:kalman1}. En la figura \ref{fig:kalman2} el robot obtiene una observación de uno de sus sensores, éste podría ser un GPS por ejemplo. Esta observación esta centrada en el pico de la gausiana representada con una línea de mayor grosor. El \textit{ancho} de esta gausiana es la incertidumbre asociada a dicha observación. La observación se incorpora al filtro y se obtiene una nueva estimación de la posición, pintada en negrita en la figura \ref{fig:kalman3}. El robot continúa desplazándose y después de un corto período de tiempo estima que su posición es la gausiana en negrita de la figura \ref{fig:kalman4}. Esta nueva estimación es más baja y ancha porque se introduce incertidumbre en la predicción de la nueva posición. Por último, en las figuras \ref{fig:kalman5} y \ref{fig:kalman6} se obtiene una nueva observación de la posición y se introduce al algoritmo para generar la nueva estimación.\\

A partir de la matriz de covarianza se puede dibujar una elipse que represente de forma gráfica el error en la muestra que representa dicha matriz. La elipse de error tiene el centro en las mismas coordenadas de la media del filtro de Kalman. Los atributos de la elipse se calculan por medio de los autovalores y autovectores. Los autovalores que se obtienen de la matriz de covarianza corresponden, cada uno de ellos, a uno de los semi-ejes de la elipse. Los autovectores nos indican la dirección de los semi-ejes. El ángulo de rotación de la elipse se calcula por medio de un simple cálculo trigonométrico a partir de uno de los autovectores. \\

Dada $A = \begin{pmatrix} a & b \\ c & d \end{pmatrix}$, que es una matriz $2x2$, $\lambda$ un número y $\hat{x}$ un vector $2x1$ con $\hat{x} \neq \hat{0}$, entonces $\lambda$ es un autovalor de $A$ y $\hat{x}$ es un autovector de $\lambda$ cuando

\begin{equation}
  A \hat{x} = \lambda \hat{x}
  \label{eq:eigenvalues}
\end{equation}

A partir de esta ecuación se pueden calcular los autovalores de la matriz de forma sencilla. Si se desarrolla un poco la ecuación anterior, tenemos
\begin{equation}
  \begin{split}
    A \hat{x} & = \lambda \hat{x} \\
    A \hat{x} - \lambda \hat{x} & = \hat{0} \\
    A \hat{x} - \lambda I \hat{x} & = \hat{0} \\
    \underbrace{( A - \lambda I )}_{\text{Soluciones no triviales} \iff det(A - \lambda I) = 0} \hat{x} & = \hat{0} 
  \end{split}
\end{equation}

El polinomio característico de la matriz es
\begin{equation}
\begin{split}
  p(\lambda) & = det(A - \lambda I) \\
             & = \begin{vmatrix} a - \lambda & b \\ c & d - \lambda \end{vmatrix} \\
             & = \lambda^2 - \underbrace{(a + d)}_{\text{diagonal}} \lambda + \underbrace{ad - bc}_{det(A)}
\end{split}
\end{equation}

Los valores obtenidos de resolver esta ecuación de segundo grado son los autovalores. Para cada uno de ellos hay que calcular su autovector correspondiente resolviendo la siguiente ecuación mediante el método que se prefiera
\begin{equation}
  \begin{pmatrix} a - \lambda & b \\ c & d - \lambda \end{pmatrix} \hat{x} = \hat{0}
\end{equation} \\

La figura \ref{fig:elipses} es una representación gráfica de las elipses de error donde se ve muy fácilmente lo que se acaba de explicar. La elipse representa el grado de incertidumbre que tiene un robot respecto a su propia localización. El robot comienza a desplazarse con una incertidumbre muy pequeña, el tamaño de la elipse es muy pequeño. Según avanza la incertidumbre crece gradualmente porque no tiene ninguna referencia visual que le ayude a verificar su posición. En el momento que divisa una baliza o \textit{landmark}, el robot es capaz de calcular su posición y la incertidumbre comienza a disminuir. En el momento que la pierde, otra vez vuelve a crecer la incertidumbre.

Esta representación es muy fácil e intuitiva de ver. De un sólo vistazo se sabe la posición del robot, o de un objeto, y su incertidumbre asociada. \\

\begin{figure} [h]
  \begin{center}
    \includegraphics[width=15cm]{img/cap3/elipses}
  \end{center}
  \caption{Representación gráfica de las elipses de error}
  \label{fig:elipses}
\end{figure}



% Capitulo 4
\chapter{Servidor de mapas dinámico}
\label{cap:sevidordemapasdinamico}


En este capítulo se explica la construcción y el funcionamiento del servidor de mapas dinámico. En primer lugar se explica que és, para que se utiliza y como se construye un \textit{costmap}. En segundo lugar se explica como se construyen los mapas que componen el algoritmo y por último se explica como se combinan estos mapas para generar el mapa usado en la navegación.

\section{Costmap\_2D}
\label{sec:costmap2d}
Un \textit{costmap} es una estructura de datos ofrecida por ROS y compuesta por un grid de ocupación y los metadatos de este grid. Cada celda del grid toma valores entre 0 y 255, donde 0 corresponde a una celda vacía, los valores entre 1 y 254 representan la probabilidad de que una celda está ocupada y el valor 255 se reserva para el desconocimiento. Cada valor se asocia con un nivel de gris, como se puede ver en la imagen.
\begin{figure} [hbtp]
  \begin{center}
    \includegraphics[width=8cm]{img/cap4/costmap-ejemplo}
  \end{center}
  \caption{Ejemplo visual de un costmap.}
  \label{fig:costmap-ejemplo}
\end{figure}\

Para poder representar la ocupación de un objeto en un \textit{costmap} es necesario hacer uso de las transformadas entre frames que nos ofrece ROS.

\subsubsection{tf}
\label{subsubsec:tf}
Cualquier robot está compuesto por multitud de piezas moviles, como puede ser la propia base del robot o la pinza de un brazo robótico. Cada una de estas piezas se pueden representar con un \textit{frame}. Ademas existen tambien otros \textit{frames} que pueden interesarnos, como puede ser el \textit{frame} de world o el \textit{frame} de map.\\
\begin{figure} [hbtp]
  \begin{center}
    \includegraphics[width=12cm]{img/cap4/frames}
  \end{center}
  \caption{A la izquierda el frame map y a la derecha los frames de odom y base\_footprint}
  \label{fig:frames}
\end{figure}\

Usamos las \textit{tf} para poder representar información relativa a uno de estos frames. Esto puede sernos de utilidad, por ejemplo, si queremos conocer la posición de un objeto que hemos cogido con nuestra pinza respecto a la base de nuestro robot, o cual es la posición relativa de un objeto que estamos percibiendo con el laser respecto a nosotros o respecto al mapa.

Cuando trabajamos con mapas es importante que todo lo que se representa en él sea respecto al frame map. De este modo nuestro mapa puede ser usado por otros nodos, como el nodo de navegación, o en cualquier otro escenario.

\section{Tipos de mapas}

En este apartado se describirá la metodologia seguida para la construcción de cada uno de los tres mapas usados por el algoritmo.

\subsubsection{Mapa estático}
El mapa estático se caracteriza por incluir las partes inmutables del escenario, como son las paredes o las puertas. La mejor manera de construirlo es medir todo el escenario y crear el mapa usando una herramienta de diseño gráfico. En este caso se ha usado \textit{Gimp}.
Este mapa nos servirá como base para crear el mapa de largo plazo.

\begin{figure} [hbtp]
  \begin{center}
    \includegraphics[width=7cm]{img/cap4/mapaestatico}
  \end{center}
  \caption{Mapa estático}
  \label{fig:mapaestatico}
\end{figure}

\subsubsection{Mapa de corto plazo}
El mapa de corto plazo se caracteriza por ser un mapa en el que se representa los objetos que el robot va percibiendo. Este mapa se inicializa con el valor 255, lo que indica una incertidumbre total. En el instante en el que el algoritmo de construcción del mapa comienza a iterar comenzarán a corregirse estos valores iniciales, asignando el valor 0 a las celdas que corresponden con zonas libres e incrementando desde 0 hasta 254 el valor de las celdas que se perciben como ocupadas. \\

{EJEMPLO CODIGO}\\

El algoritmo propuesto destaca por la capacidad de no solo añadir objetos al mapa, si no ademas eliminarlos si los objetos desaparecen del lugar que ocupaban. Para ello se compara cada muestra de datos con el mapa que estamos generando y si en dicha muestra existen celdas libres que en el mapa están ocupadas se decrementa el valor de dicha celda en el mapa. La cuantia del decremento se puede modelar, consiguiendo asi que el robot olvide más lentamente o más rapidamente los objetos que desaparecen del escenario.\\

{EJEMPLO CODIGO}\\
{EJEMPLO GRAFICO}\\




\section{Colección de EKFs}
\label{sec:colecciondeekfs}

Una vez explicada la adaptación del Filtro Extendido de Kalman, ahora hay que integrarlo con el resto del sistema. El EKF, por sí solo, únicamente permite hacer el seguimiento de un objeto. Para poder hacer el seguimiento de varios objetos simultáneamente se utiliza un conjunto de EKFs, cada uno de ellos asociado a uno de los objetos. El conjunto de filtros es dinámico pudiendo ajustarse un número mínimo de observaciones que siempre estarán presentes. En el caso de la pelota es uno, mientras que en el caso de las porterías es dos, poste izquierdo y derecho.\\



Para poder manejar dinámicamente el conjunto de filtros es necesario definir una serie de reglas para añadir un nuevo filtro o para actualizar/eliminar uno de los ya existentes. Hay que tener en cuenta que los filtros creados en el algoritmo puede que no representen un objeto real del entorno, sino que sea un falso positivo. En estos casos, lo ideal es detectarlo cuanto antes y eliminarlo.\\

Realizar el seguimiento de varios objetos es fácil si los objetos no son homogéneos, es decir, si tenemos un método capaz de distinguir cada uno de ellos. Por ejemplo, una pelota azul y una roja. El problema viene cuando realizar la distinción entre varios objetos no es posible. Como por ejemplo pueden ser las lecturas de un radar o los postes de las porterías. El pseudocódigo del algoritmo que se propone en esta investigación se puede ver en Algoritmo \ref{alg:seguimientoobjetos}. Éste puede dividirse en cinco fases:

\begin{packed_item}
\item La instrucción 1 representa la inicialización del algoritmo y se ejecutan una sola vez, al iniciar el algoritmo. Las instrucciones entre las líneas 2 y 22 se ejecutarán en bucle durante todo el período de vida del algoritmo.
\item Entre 2 y 6 se predice el nuevo estado de los objetos que están siguiendo.
\item En las líneas 7 y 14 se incorporan las nuevas observaciones al sistema, ya sea actualizando alguno de los objetos ya existentes o bien, si ninguno satisface las condiciones, creando uno nuevo.
\item Las líneas 15 y 17 añaden ruido a los objetos sospechosos de ser falsos positivos.
\item Por último, entre las líneas 18 y 22 se eliminan los objetos que tengan demasiada incertidumbre.
\end{packed_item}

Además del estado y la incertidumbre de los objetos de seguimiento se guardan algunos datos más que los enriquecen y permite tomar mejores decisiones. Se guarda la marca de tiempo en la que se predijo por última vez el nuevo estado y la marca de tiempo de la última vez que se vio el objeto. También se guarda el \textit{origen} de la observación, que puede ser local o remota. En este proyecto todas las observaciones que se utilizan son locales, pero ésto deja la puerta abierta a futuras mejoras, que se explicarán en el último capítulo. A continuación se detallan cada una de las fases del algoritmo y las operaciones que se realizan en ellas.\\

\subsection{Inicialización de los filtros}
\label{subsec:inicializacion}

La inicialización del algoritmo sólo se produce la primera vez. El algoritmo permite definir un número mínimo de instancias que siempre estarán presentes en la colección. Esta propiedad es útil cuando el sistema donde se integra el algoritmo no permite valores nulos. Un ejemplo sencillo es el caso de la pelota en un partido de la RoboCup. Se sabe que la pelota está siempre presente en el campo. Puede que no se conozca su localización o que esté oculta por el resto de jugadores, pero la pelota está en el terreno de juego. En este caso tiene sentido tener como mínimo una instancia de la pelota en nuestro sistema. En el caso de no haber visto la pelota en mucho tiempo, la instancia tiene almacenada la última posición donde se vio, pero la incertidumbre será tan alta que permite discernir si esa posición es fiable o no.\\

Es necesario inicializar estas instancias con unos valores arbitrarios. La posición inicial de estas instancias no es importante, lo único que hay que tener cuidado es de inicializar la incertidumbre con un valor suficientemente alto para que la instancia no sea fiable y el robot pueda actuar en consecuencia, lo que sería siguiendo la línea del ejemplo anterior, iniciando la búsqueda de la pelota. \\

Otro momento en el que hay que inicializar filtros es cuando tenemos una observación nueva que no corresponde a ninguno de los filtros presentes. Esta operación se describe en \ref{subsubsec:Matching}. En este caso, la inicialización es más sencilla. El filtro se inicializa directamente con los datos de la observación.

\subsection{Predicción de nuevos estados}
\label{subsec:prediccionnuevosestados}

En esta fase, instrucciones entre las líneas 2 y 6, se predice el nuevo estado para cada uno de los objetos sobre los que se está realizando el seguimiento. El estado y la incertidumbre se actualizan mediante la fase de predicción del EKF que ya se ha explicado anteriormente. También se actualiza la marca de tiempo en la que se realiza la predicción.\\

\begin{figure}[h]
  \centering
  \subfloat[El robot está siguiendo dos objetos y se mueve en dirección hacia ellos]{
    \label{fig:prediccion_01}
    %\includegraphics[width=7cm]{img/cap4/prediccion_01}
  }
  \subfloat[Los filtros se encuentran más cerca del robot y ha aumentado su incertidumbre]{
    \label{fig:prediccion_02}
    %\includegraphics[width=7cm]{img/cap4/prediccion_02}
  }
  \caption{Comportamiento de los filtros en la fase de predicción}
  \label{fig:prediccion}
\end{figure}

La figura \ref{fig:prediccion} muestra el comportamiento de los filtros en esta fase. En la imagen de la izquierda el robot está realizando el seguimiento de dos objetos. Cada uno tiene una posición y una incertidumbre. El robot se encuentra moviéndose en dirección a ellos. En la imagen de la derecha se puede ver el resultado de la fase de predicción. Los objetos se encuentran ahora en una posición más cercana al robot, hay que tener en cuenta que las posiciones de los objetos son relativas a éste, y la incertidumbre de los filtros aumenta en consecuencia. \\

\subsection{Incorporación de observaciones}
\label{subsec:incorporacionobservaciones}

En esta fase se realizan varias operaciones para decidir si actualizar un filtro ya existente o, si ninguno satisface las condiciones, crear uno nuevo. Para ello se emparejan las observaciones con los filtros y se calcula el valor de similitud de ellos. Seguidamente se escoge la pareja que más se parezca y, si se cumplen las condiciones necesarias, se actualiza el filtro con la observación. En caso de no cumplirse las condiciones, se crea un filtro nuevo a partir de la observación. Se repite este procedimiento con todas las observaciones. \\

\subsubsection{Emparejamiento}
\label{subsubsec:emparejamiento}

El criterio de emparejamiento utilizado en nuestro algoritmo se basa en la posiciones e incertidumbres tanto de la observación como de cada uno de los filtros. Vincula una observación a uno de los filtros. Para este criterio utilizamos la función de \textit{verosimilitud}, \ref{eq:likelihood}. Esta función permite realizar inferencias acerca del valor de los parámetros de un modelo estadístico a partir de un conjunto de observaciones. En este paso se compara la observación obtenida con todos los filtros y se escoge el que tenga mayor similitud. $s1$ y $P1$ son el estado y la matriz de covarianza de un filtro, respectivamente, y $s2$ y $P2$ representa lo mismo pero de la observación.

\begin{equation}
\label{eq:likelihood}
likelihood = \frac{ e^{ det(-\frac{1}{2} (s1-s2)^T (P1+P2)^I (s1-s2) )} }
                  { \sqrt{ det(2\pi(P1+P2)) } }
\end{equation}

El estado y la matriz de covarianza de cada uno de los filtros no tiene el tamaño original de $4x1$ y $4x4$. Antes de utilizarse en la ecuación se simplifican las matrices eliminando los componentes de velocidad $Vx$ y $Vy$ para que el tamaño de las matrices coincida con los tamaños de las observaciones y pueda aplicarse la ecuación.\\

\begin{figure}[h]
  \centering
  \subfloat[Fusión de una observación con un filtro]{
    \label{fig:emparejamiento_01}
    %\includegraphics[width=7cm]{img/cap4/emparejamiento_01}
  }
  \subfloat[Posible creación de un filtro nuevo]{
    \label{fig:emparejamiento_02}
    %\includegraphics[width=7cm]{img/cap4/emparejamiento_02}
  }
  \caption{Distintas situaciones del emparejamiento entre una observación y un filtro}
  \label{fig:emparejamientos}
\end{figure}

En la figura \ref{fig:emparejamientos} se puede ver un ejemplo gráfico más fácil de entender. En las imágenes aparecen una colección de tres filtros y dos situaciones distintas al incorporar dos observaciones distintas. En la figura \ref{fig:emparejamiento_01} se ve claramente que la observación se emparejaría con el filtro 1. En la figura \ref{fig:emparejamiento_02} la situación no está tan clara. La observación se encuentra prácticamente a la misma distancia de todos ellos. La observación se emparejaría con el filtro 3 por ser el que más incertidumbre tiene.

\subsubsection{Matching}
\label{subsubsec:Matching}

Una vez que se ha escogido uno de los filtros existentes hay que decidir si actualizarlo con la nueva observación o crear una filtro nuevo. El método que utilizamos es bastante simple e intuitivo. Se comprueba si las elipses generadas a partir de la media y covarianza de la observación y del filtro intersectan, en cuyo caso se considera que ambas se corresponden con el mismo objeto y se actualiza el filtro con la observación. En caso contrario, se crearía un nuevo filtro. \\

Al calcular la intersección de las elipses, el tamaño de éstas se multiplica por 4 siguiendo la desigualdad de Chebyshev. Este factor es un resultado que ofrece una cota inferior a la probabilidad de que el valor de una variable aleatoria con varianza finita esté a una cierta distancia de su esperanza matemática. En la siguiente ecuación, se puede ver el valor de la desigualdad de Chebyshev para un valor $k$ veces la desviación típica partiendo de la media. En nuestro caso, hemos escogido el valor 4 y devuelve un resultado muy próximo a $94\%$. \\

\begin{tabular}{cc}
  \textbf{Población mínima} & \textbf{\# de desviaciones típicas desde la media} \\
  50\% & $\sqrt{2}$ \\
  75\% & 2 \\ 
  89\% & 3 \\
  94\% & 4 \\
  96\% & 5 \\
  97\% & 6 \\
  $1-\frac{1}{k^2}$ & k \\
  l & $\frac{1}{\sqrt{1-l}}$
\end{tabular}
\\

Esta tabla se puede leer: para cualquier distribución con una desviación estándar definida, la cantidad de datos que se encuentran a una distancia $k$ veces la desviación típica de la media es, al menos, como se indica en la tabla. Es decir, a una distancia 4 veces la desviación típica partiendo de la media se encuentran el 94\% de los datos. \\

La intersección de dos elipses no es un cálculo sencillo. Hay que tener en cuenta distintos casos: cuando la distancia entre las elipses es demasiado grande para intersectar, cuando el centro de una ellas está contenida dentro de la otra que intersecta siempre, o cuando simplemente intersectan en uno, dos, tres o cuatro puntos. Las elipses están definidas por 5 parámetros: $x$ e $y$, que son las coordenadas del centro; $a$  $b$, que son los semi-ejes mayor y menor respectivamente, y $\lambda$ que es la rotación. \\

\begin{enumerate}

\item Primero se comprueba que las elipses estén a una distancia mínima en la puedan intersectarse o una esté contenida dentro de la otra. En caso de no cumplirse este requisito, se puede afirmar que las elipses no intersectan. Por lo que si la distancia entre los centros es mayor que la suma de los dos semi-ejes de mayor tamaño, las elipses no intersectan.
\begin{equation}
  \sqrt{ (x_1 - x_2)^2 + (y_1 - y_2)^2 } > a_1 + a_2 \rightarrow \text{no intersectan}
\end{equation}

\item Si el centro de una de las elipses está contenido dentro de la otra, entonces se puede afirmar que las elipses intersectan o una está contenida totalmente dentro de la otra, situación igualmente válida. $p$ y $q$ son las coordenadas del centro de la otra elipse.
\begin{equation}
  \frac{ ((p \cos \lambda + q \sin \lambda ) - x)^2 }{ a^2 } + 
  \frac{ ((p \sin \lambda - q \cos \lambda ) - y)^2 }{ b^2 } \leq 1
\end{equation}

\item Por último, si en caso de no poder afirmar nada aún hay que resolver los puntos de intersección de las elipses. A partir de las ecuaciones de la elipse en su forma
\begin{equation}
  \begin{split}
  ax^2  + bxy  + cy^2  + dx  + ey  + f = 0 \\
  a'x^2 + b'xy + c'y^2 + d'x + e'y + f' = 0    
  \end{split}
\end{equation}

se obtiene una ecuación de cuarto grado que puede resolverse en forma cerrada. Esto quiere decir que se puede resolver en términos de funciones y operaciones matemáticas elegidas de un conjunto limitado. Una ecuación de cuarto grado tiene la siguiente forma
\begin{equation}
  ax^4 + bx^3 + cx^2 + dx + e = 0
\end{equation}

Se pueden hacer una serie de comprobaciones rápidas en la ecuación para averiguar tiene soluciones no triviales.
\begin{itemize}
\item Si $e = 0$ una de las raíces es $0$, por lo tanto intersectan por lo menos en un punto.
\item Si $a=0$, $b \neq 0$, se trata de una ecuación cúbica que siempre tiene raíces.
\item Si $a=0$, $b=0$, $c \neq 0$, ecuación cuadrática. Sencillo de resolver mediante la fórmula cuadrática $x=\frac{-b \pm \sqrt{b^2 - 4ac}}{2a}$
\item Si $a=0$, $b=0$, $c=0$ y $d \neq 0$, todas las rectas con pendiente distinta de 0 tienen una raíz
\end{itemize}

En cualquier otro caso hay que resolver la ecuación de cuarto grado, operación más costosa computacionalmente.

\end{enumerate}

\subsection{Descarte de filtros}
\label{subsec:descartefiltros}

Para obtener un rendimiento mayor es necesario tener el menor número de filtros posibles ejecutándose al mismo tiempo. En caso de no borrarse ningún filtro, al ser una colección dinámica, el tamaño podría crecer desmesuradamente en un corto espacio de tiempo perjudicando enormemente el rendimiento del algoritmo. Normalmente se quiere eliminar un filtro cuando se trata de un falso positivo o cuando un objeto sobre el que se estaba realizando el seguimiento desaparece del entorno.\\

La colección de filtros tiene un número mínimo de objetos sobre los que hace el seguimiento. Este límite es útil para situaciones en las que se puede afirmar que siempre estará presente en el entorno ese mínimo número de objetos, como, por ejemplo, cuando se realiza el seguimiento de una pelota en un partido de la RoboCup. Siempre hay una pelota, aunque el robot no la haya localizado o la haya perdido de vista. En el caso de haber localizado en algún momento la pelota, pero haberla perdido de vista hace un tiempo, el filtro almacena la última posición conocida de la pelota. Al haberse perdido de vista hace tiempo la pelota tendrá una incertidumbre alta. Esta información es muy útil para comportamientos de más alto nivel ya que pueden basar sus decisiones, además de en la posición de la pelota, en su incertidumbre.\\

El método utilizado es bastante simple. Si hay un número mayor de filtros que el número mínimo especificado, se compara el área de la elipse con un umbral determinado y, en caso de superarse, el filtro se elimina. Este umbral se elige de forma arbitraria. El método usado para que sea sencillo elegir el umbral es crearlo a partir del área de una circunferencia. Dependiendo del modelo del objeto que se utilice se puede tener un umbral más alto o más bajo, pudiendo adaptarse el algoritmo a mayor cantidad de problemas.

\begin{equation}
  \underbrace{2ab}_{\text{Área elipse}} > \underbrace{\pi r^2}_{\text{Área circunferencia}} \rightarrow \text{eliminar filtro}
\end{equation}

\section{Adición de incertidumbre}
\label{sec:adicionincertidumbre}

El algoritmo del Filtro de Kalman está pensado para tener una fuente de datos continua. En cada iteración se actualiza la información del filtro con las nuevas observaciones. El problema es que normalmente un robot tiene una visión parcial del problema y puede no percibir todos los objetos todo el tiempo. El algoritmo básico de Kalman no contempla esta situación. La única manera de aumentar la incertidumbre es mediante el movimiento del robot o mediante otras observaciones. Pero, ¿qué pasa cuando un robot está parado y no detecta un objeto? La forma más sencilla de entenderlo es mediante un ejemplo. \\

Durante un partido de fútbol suponemos que el portero tiene localizada la pelota con muy poca incertidumbre. En ese momento se empiezan a juntar jugadores alrededor de ella, provocando que el portero la pierda de vista. Si el portero está estático, al no tener ninguna observación de la pelota, el filtro mantiene el mismo estado y la misma incertidumbre a lo largo del tiempo. En ese período de tiempo lo más probable es que la pelota se haya movido del sitio en el que estaba, pero el portero sigue con una incertidumbre muy baja lo que le lleva a tomar decisiones como si la pelota siguiese en ese sitio. Decisiones probablemente desacertadas. \\

Otro comportamiento que se ha observado que no favorece el seguimiento de objetos con movimiento se produce cuando se observa un objeto durante un período de tiempo seguido en el mismo sitio. El filtro tiende a disminuir demasiado la incertidumbre provocando que no sea reactivo a nuevos movimientos del objeto. En nuestra implementación en concreto, se generaba un nuevo filtro para el mismo objeto. \\

En el primero de los casos, el comportamiento buscado es aumentar la incertidumbre del objeto cuando éste no se visualice. De esta manera, el portero, podría reaccionar frente a la situación de perder de vista la pelota entre los jugadores. En todo momento sigue recordando la última posición en la que ésta se encontraba, pero es capaz de reaccionar mucho antes a las nuevas situaciones que se presenten. Por otro lado, en el segundo caso, también se busca introducir un poco de ruido en los filtros para mantenerlos más reactivos, sobre todo aquellos que tienen más movimiento. \\

Se han definido tres niveles de adición de incertidumbre distintos dependiendo de la situación:
\begin{enumerate}

\item El filtro se ha actualizado con una observación. En este caso se añade un mínimo de ruido para mantener el filtro reactivo. La cantidad de incertidumbre depende de los valores con los que se ha modelado el objeto.

\item El filtro no se ha actualizado porque no se encuentra en el \textit{frustum} de la cámara. De un objeto en esta situación no se puede afirmar nada, si sigue en el mismo lugar o se ha desplazado, por lo que se añade una cantidad \textit{estándar} de ruido.

\item Si un filtro no se actualiza y debería estar en la imagen. Esta situación la tomamos como un falso positivo. Puede que en realidad el filtro no sea un falso positivo y sea una oclusión o que el detector no lo localice en la imagen, pero es muy difícil discernir entre estas posibilidades.

\end{enumerate}

En esta fase se decide en cuál de las tres situaciones se encuentra cada uno de los filtros y se suma el ruido correspondiente a la matriz de covarianza del filtro:
\begin{equation}
  S' = S + noise^{\# \text{situación}} I
\end{equation}

\section{Objetos complejos}
\label{sec:objetoscomplejos}

En ocasiones se quieren seguir objetos que están compuestos de formas más simples y más sencillas de reconocer. Esto se hace porque el objeto es demasiado grande y no puede ser detectado con una sola imagen, o bien porque es demasiado complejo para resolver el problema de una vez y se divide en problemas más sencillos. \\

En el caso de las porterías en el fútbol se dan los dos casos expuestos. Detectar una portería al completo es una tarea bastante difícil, ya que habría que detectar los dos postes al mismo tiempo para poder detectar que ese objeto es una portería. Pero los postes están separados por una distancia considerable y dependiendo de la distancia a la que se encuentre el robot de éstos, puede ser imposible ver los dos postes al mismo tiempo. Por esto se ha simplificado el problema inicial en hacer el seguimiento de los postes y, a partir de estos, averiguar los dos postes con más probabilidades de ser una portería.

\begin{figure} [h]
  \centering
  %\includegraphics[width=15cm]{img/cap4/prod_vectorial}
  \caption{Distinguir poste izquierdo de poste derecho}
  \label{fig:prod_vectorial}
\end{figure}

Para ello se crea una colección de filtros con un mínimo de dos instancias de los postes de la portería. Simplemente se siguen los postes sin tener en cuenta si son de una portería o no. \textit{A posteriori} se realiza una serie de cálculos para, de entre todos los postes de la colección, decidir cuáles son más probables de ser una portería. Se realiza la siguiente operación a cada uno de las combinaciones de postes que haya en la colección. $s$ es la posición del poste y $P$ su covarianza.
\begin{equation}
  \begin{split}
    s' = f(s_1 - s_2) \\
    P' = A (P_1 + P_2) A^T 
  \end{split}
\end{equation}

donde $f()$ es la función para pasar del espacio euclídeo $\mathbb{R}^2 \rightarrow \mathbb{R}^1$ y $A$ es la matriz jacobiana de la función
\begin{equation}
  \begin{split}
    f(s) = \sqrt{s_x^2 + s_y^2} \\
    A = \begin{pmatrix} \frac{s_x}{\sqrt{s_x^2 + s_y^2}} & \frac{s_y}{\sqrt{s_x^2 + s_y^2}} \end{pmatrix}    
  \end{split}
\end{equation}

Una vez hecho esto, para cada par de postes de la colección se tiene la distancia entre ellos y la covarianza de ésta. Para decidir qué par de postes es el que más se parece a una portería, se utiliza el mismo método que cuando se emparejan las observaciones con los filtros, se calcula de la distancia de \textit{Mahalanobis}. 

%Poner flecha de vector a x e y.
\begin{equation}
  d(x,y) = \sqrt{(x-y)^T S^{-1} (x-y)}
\end{equation}

Esta distancia es un método estadístico descriptivo que aporta una medición relativa de varios puntos a uno común. El valor calculado no tiene unidades de medida, lo que lo convierte en un método muy útil para comparar distancias teniendo en cuenta la incertidumbre. En nuestro caso se calcula el error de distancia entre los dos postes, $1500 - dist(post1,post2)$. \\

Una vez escogidos los postes que van a representar la portería, hay que diferenciar el izquierdo del derecho. Se calcula mediante el producto vectorial entre los dos postes. Si el producto entre el poste 1 y el poste 2 es positivo, entonces el poste 1 es el derecho y el poste 2 es el izquierdo. Si el producto es negativo, es a la inversa. La figura \ref{fig:prod_vectorial} muestra el cálculo.


% Capitulo 5
\chapter{Aplicaciones}
\label{cap:aplicaciones}

En este capítulo se enumeran y describen las aplicaciones que usarán el mapa que construimos dinamicamente. Estas aplicaciones serán, principalmente, el nodo de localización y el nodo de navegación.

\section{AMCL}
\label{sec:amcl}
\textit{AMCL}\footnote{http://wiki.ros.org/amcl} es un paquete de localización que implementa un algoritmo de Monte Carlo, el cual usa un filtro de partículas para localizar al robot sobre un mapa que previamente le proporcionamos.
\subsubsection{Filtro de partículas}
El algoritmo del filtro de partículas se divide en 4 etapas: Inicialización, actualización, estimación y predicción. 
En la etapa de inicialización se ``lanzan'' una serie de partículas cercanas a la posición inicial del robot. Estas partículas ademas de una posición en el espacio tambien tendrán una dirección. Podemos ver un ejemplo gráfico en la imagen \ref{fig:initamcl}

\begin{figure}[hbtp]
  \begin{center}
    \includegraphics[width=10cm,height=7cm]{img/cap5/initamcl}
  \end{center}
  \caption{Inicialización del filtro de partículas}
  \label{fig:initamcl}
\end{figure}
Vemos como se han generado muchas partículas alrededor del robot y que cada una tiene una dirección más o menos acertada con la dirección del robot.

En este punto del algoritmo se compara la percepción del laser del robot en el punto en el que se encuentra con la percepción que tendria si tuviera la posición y la dirección de cada una de las partículas que generamos. Cuanto más acertada sea la suposición anterior, más valor se le dá a esa partícula. Asi nos encontraremos que las partículas que estan más cercanas a la posición del robot cobran más valor y las que están más lejos y con una dirección totalmente erronea tienen menos valor. Esta fase es la llamada de Actualización. 

En la siguiente fase, llamada de Estimación, nos quedamos con las partículas que más valor tenian para volver a lanzarlas en la siguiente fase del algoritmo.

En la ultima fase, fase de Precicción, lanzamos las partículas de nuevo con el valor que tenian y su posición, añadiendole un pequeño ruido.
En este punto del algortimo tambien se corrige la posición del robot a la posición de la partícula con más valor.
Una vez completado el algoritmo se vuelve a la fase de Actualización y se repite hasta que el robot esté perfectamente localizado en el 
mapa.

\begin{figure}[hbtp]
  \begin{center}
    \includegraphics[width=10cm,height=7cm]{img/cap5/actamcl}
  \end{center}
  \caption{El robot se vá acercando a la posición ideal}
  \label{fig:actamcl}
\end{figure}
\pagebreak

Observamos como la linea color de la parte derecha de la imagen \ref{fig:actamcl}, correspondiente a las muestras tomadas con el laser, está más cerca de la linea del mapa que en la imagen \ref{fig:initamcl} que se aprecia que no está alineada. Esto es fruto de la corrección que se va haciendo de la posición. Tambien observamos que hay menos partículas, esto es fruto de la convergencia hacia la posición correcta.

\begin{figure}[hbtp]
  \begin{center}
    \includegraphics[width=10cm,height=7cm]{img/cap5/finamcl}
  \end{center}
  \caption{El robot se encuentra totalmente localizado}
  \label{fig:finamcl}
\end{figure}

En la imagen \ref{fig:finamcl} vemos como el número de partículas se ha reducido mucho, ya que se ha llegado a casi una estimación de la posición del robot muy cerca de la posición real. Vemos tambien que las lineas de color correspondientes a las muestras del laser están alineadas con el mapa.

Para el uso del paquete \textit{amcl} en nuestro algoritmo fué necesaria la realización de una pequeña modificación. Esta modificación se refiere a que el paquete por defecto solo usa un mapa y lo obtiene al principio de la ejecución del algoritmo. Si le llegaba un nuevo mapa reiniciaba por completo el algoritmo. Esto nos generaba un problema, ya que en el algoritmo propuesto se publica un mapa por cada iteración y el paquete por defecto se reiniciaba constantemente. El efecto que producía es que el robot siempre se encontraba en la posición inicial y aunque lo movieramos siempre ocupaba la misma posición en el mapa. En nuestro \textit{amcl} modificado se usa el mapa que se obtiene en cada iteración y sobre el se calcula la posición del robot, sin reiniciar en ningún momento el algoritmo.

Además esta modificación nos ha permitido localizarnos en entornos desconocidos, es decir, ahora contamos con la capacidad de navegar por estancias de una casa que no tenemos en el mapa. Estas estancias se irán añadiendo al mapa de corto plazo primero y más tarde al mapa de largo plazo y el algoritmo de localización puede seguir calculando nuestra posición en los nuevos mapas.

% Capitulo 6
\chapter{Conclusiones y trabajos futuros}
\label{cap:conclusiones}

Una vez presentados los detalles del algoritmo desarrollado, su validación y verificación, en este último capítulo se exponen las conclusiones y se hace autocrítica del trabajo. Para finalizar, se proponen una serie de mejoras y ampliaciones que mejoren el comportamiento del algoritmo.

\section{Conclusiones}
\label{sec:conclusiones}

En este proyecto se ha desarrollado un algoritmo para hacer un seguimiento de los objetos más importantes del entorno de un robot. Como se ha explicado en capítulos anteriores, es muy importante filtrar la información recibida por parte de los sensores. En el caso de la cámara, la información se filtra dos veces. Una primera vez para detectar el objeto en cuestión en la imagen y otra para filtrar esa información y, en la medida de lo posible, reducir al máximo el ruido existente en las observaciones. De esta manera se consiguen resultados más precisos y robustos, lo que hace que el robot tome mejores decisiones. \\

Como conclusión general del proyecto se puede decir que se han cumplido satisfactoriamente los requisitos y objetivos marcados en el capítulo \ref{cap:objetivos}. El estado de las instancias del algoritmo está formado por la posición y la velocidad del objeto. Aunque la velocidad no es directamente medible, ésta se obtiene por inferencia a partir de las posiciones previas, obteniendo una medida bastante acertada. La colección de filtros es dinámica. Puede seguir varios objetos al mismo tiempo y, aunque sean indistinguibles para el detector, el algoritmo es capaz de emparejar las observaciones con su filtro correspondiente. El algoritmo se ejecuta en tiempo real y no consume muchos recursos. \\

La datos de entrada del algoritmo no están ligados a ningún sensor o detector concreto. El único imprescindible para ejecutarlo es la posición de un objeto en un instante determinado del tiempo. Igualmente tampoco influye en el funcionamiento del algoritmo si los objetos son dinámicos o estáticos. Si el comportamiento de estos está bien modelado, tendremos unos resultados precisos. \\

La implementación del algoritmo es independiente de la plataforma, es decir, en el código del núcleo de ésteno hay ninguna referencia a componentes dependientes del sistema donde se ha desplegado. Esto facilita enormemente su exportación y podría, incluso, compilarse como una biblioteca independiente y ser usado en cualquier proyecto. Se ha intentado diseñar el algoritmo de forma que sea flexible y fácilmente extensible, ya que cada situación se resuelve de una manera distinta. Por último, la precisión y robustez de los algoritmos de las observaciones se ha mejorado. Los datos son más fiables y estables que si se operase con ellos directamente desde el detector. \\

El desarrollo del proyecto se ha llevado a cabo mediante pequeñas iteraciones en las que se han ido implementando cada una de las partes del algoritmo. Se ha puesto especial énfasis en la programación de las operaciones matemáticas. Muchas de las ecuaciones son bastante largas y complejas, hecho que provoca muchos pequeños fallos a la hora de escribir el código. Un simple cambio de signo de un término de la ecuación suele traducirse en comportamientos inesperados y extravagantes.

\section{Trabajos futuros}
\label{sec:trabajosfuturos}

El trabajo desarrollado en este proyecto ha dejado la puerta abierta a multitud de mejoras y nuevos trabajos que se pueden hacer partiendo de éste. Esto es sólo la punta del iceberg. En cuanto al algoritmo se puede mejorar lo siguiente:

\begin{itemize}

\item Incorporar aceleración al estado de los objetos. El modelo actual sólo contempla la posición y velocidad de los objetos. La aceleración del objeto ayudaría a realizar una mejor predicción del nuevo estado. Esto puede ser de especial interés en entornos más complejos que el del fútbol robótico. Por ejemplo, un coche autónomo podría medir la aceleración de los coches de su alrededor para predecir con mayor precisión futuros estados y tomar mejores decisiones. 

\item Modelo de movimiento de los objetos más preciso. El modelo actual es completo pero simple. El rozamiento del objeto con el suelo está modelado mediante un simple parámetro que disminuye la velocidad del objeto en cada iteración. A partir de la mejora anterior, una opción es modelar el rozamiento como una aceleración negativa a la velocidad.

\item Comparación con UKF. El UKF es otra variante del algoritmo de Kalman para sistemas no-lineales. Utiliza otra técnica distinta de linealización que el EKF. En los últimos años está cobrando bastante popularidad. La linealización es menos engorrosa ya que no hay que calcular las derivadas simples ni Jacobianas de las ecuaciones de predicción y corrección. Sería interesante comparar el rendimiento de ambos y ver cuál tiene mejores resultados.

\item Mejorar operación de emparejamiento. El emparejamiento es una operación de complejidad $O(n^2)$. El tiempo de resolución del problema crece exponencialmente con el tamaño de los filtros y observaciones que se tengan. Para valores pequeños de $n$ no se nota mucho, pero según crece el número de instancias a solucionar, la resolución crece enormemente.

\end{itemize}

En el campo de las aplicaciones directas que tiene el algoritmo se proponen los siguientes proyectos.

\begin{itemize}

\item Cálculo de trayectorias en un portero. Teniendo la velocidad de un objeto es bastante sencillo calcular la trayectoria de este. Un comportamiento muy interesante para un portero sería la de calcular esta trayectoria para lanzarse hacia alguno de los lados para parar la pelota. Basta con averiguar si la pelota va a pasar por el lado izquierdo o derecho del robot. 

\item Cálculo de trayectorias en un jugador. El cálculo de la trayectoria de la pelota también es una función interesante para un jugador. El robot puede calcular la futura posición de la pelota e ir directamente a esa posición para llegar más rápido que los contrarios. En la figura \ref{fig:planning} se presenta un pequeño diagrama que representa esta idea.

\begin{figure}[h]
  \centering
  \subfloat[Trayectoria parabólica]{
    \label{fig:planning01}
    \includegraphics[width=7cm]{img/cap6/planning01}
  }
  \subfloat[Trayectoria rectilínea]{
    \label{fig:planning02}
    \includegraphics[width=7cm]{img/cap6/planning02}
  }
  \caption{Diferencia de trayectoria con/sin tener en cuenta la velocidad de la bola}
  \label{fig:planning}
\end{figure}

En la figura \ref{fig:planning01} se puede ver la típica trayectoria que toma un jugador que sólo conoce las posiciones instantáneas de la pelota. Es una trayectoria con forma de parábola. Según se acerca el jugador a la pelota, está se va moviendo y el robot siempre toma las decisiones con retraso. En cambio, si se conoce la posible posición final de la pelota -figura \ref{fig:planning02}- , se puede ir directamente a esta localización y utilizar el sistema de visión, simplemente para comprobar que la hipótesis es correcta y, al mismo tiempo, realizar pequeños ajustes.

\item Seguimiento de otros jugadores. El algoritmo se puede utilizar también para realizar el seguimiento de los otros robots que están en el campo. Esto puede ayudar enormemente en el cálculo de caminos para evitar a los jugadores con antelación.

\item Estado compartido de la pelota. Se pueden introducir observaciones de otros jugadores para intentar mejorar la percepción de la pelota, llegando incluso al punto de conocer la posición de la pelota sin ni siquiera verla.

\end{itemize}


% Bibliografía
\clearpage
\addcontentsline{toc}{chapter}{Bibliografía}
\nocite{*}	% Imprime todas las cites de la bibliografía aunque no se citen
\bibliographystyle{named} 
\bibliography{bibliografia}

\printindex 

\end{document}
